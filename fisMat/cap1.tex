\section{Meccanica Newtoniana}
\subsection{L'equazione di Newton}
\begin{equation}
    f = m a 
\end{equation}
Il \textit{secondo principio della dinamica} di Newton è l'equazione del moto di una particella di massa $m$ 
soggetta a una forza $f$, $a$ è l'accelerazione.\\ 
\begin{equation}
    m \ddot{x} = f( x(t),\dot{x}(t), t) \qquad \text{ ove } \quad v = \dot{x}(t) = \dv{x}{t}\quad a = \ddot{x}(t)= \dv[2]{x}{t}
\end{equation}
La forza $f$ è una funzione assegnata del tipo $f: \mathbb{R}^d\times \mathbb{R}^d \times \mathbb{R} \rightarrow \mathbb{R}^d$\\
L'equazione di newton è un'\textit{equazione differenziale ordinaria} ODE di secondo ordine con incognita $x(t)$. 
In base ai dati iniziali e alla funzione $f$ possono valere o meno i teoremi di esistenza e unicità $\exists!$ locale e globale.\\ 
$x(t)$ è una curva del tipo $I \subseteq \mathbb{R}\quad x: I \rightarrow \mathbb{R}^d $. L'equazione può essere 
riscritta nella forma di sistema equivalente:
\begin{equation}
    \begin{cases}
        \dot{x}(t) = v(t)\\ 
        \dot{v}(t) =\frac{1}{m} f( x(t), v(t), t)
    \end{cases}
    \implies
    \qquad
    \dot{X}(t)= F(t)
\end{equation}
ove:
\begin{equation*}
    X (t) = \begin{pmatrix}
        x (t) \\ 
        v (t)
    \end{pmatrix} \subseteq \mathbb{R}^d \times \mathbb{R}^d \simeq \mathbb{R}^{2d} \qquad
    F (t) = \begin{pmatrix}
        v(t) \\
        f(x(t), v(t),t)
    \end{pmatrix}
\end{equation*}

Il moto può essere univocamente determinato se assegno il dato iniziale $x(t_0)=x_0,\;\dot{x}(t_0)= v(t_0)=v_0$.\\
Conoscendo i dati iniziali, cerchiamo una soluzione locale tramite sviluppo di Taylor: 
\begin{equation*}
    x(t_0) \text{ noto}\implies x(t_0+h)= x(t_0)+\dot{x}(t_0) h +\frac{1}{2!}\ddot{x}(t_0) h^2 +\frac{1}{3!}\dddot{x}(t_0)h^3  + o(h^4)= 
\end{equation*}
\begin{equation*}
   = x_0 +v_0 h + \frac{f}{2m}(x_0,v_0,t_0) h^2 + \dots
\end{equation*}
Si può scrivere:
\begin{equation}
    \begin{cases}
        \dd{x}= v(t)\dd{t}\\
        \dd{v}= \frac{1}{m}f(x(t),v(t),t)\dd{t} 
    \end{cases}
\end{equation}
\begin{equation*}
    (x,v) \text{ in } t \implies  \text{ in } t +\dd{t} \quad (x',v')= (x+\dd{x},v+\dd{v})= (x+v\dd{t},v+ \frac{f}{m}\dd{t})
\end{equation*}

\begin{definition}
    \begin{equation*}
        \begin{pmatrix}
            x\\v  
        \end{pmatrix}\in \mathbb{R}^{2d} 
    \end{equation*}
    è detto \textit{spazio delle fasi} del sistema. 
\end{definition}

\begin{definition}
    La curva soluzione $t\rightarrow(x(t),v(t))$ si chiama \textit{curva di fase}, distinta da $t\rightarrow x(t)$ che si chiama 
    \textit{traiettoria o orbita}.
\end{definition}

\begin{definition}
    \begin{equation*}
        F=\begin{pmatrix}
            v\\\frac{1}{m}f 
        \end{pmatrix}
    \end{equation*}
    si chiama \textit{campo vettoriale newtoniano} ed è geometricamente il vettore tangente alla curva di fase passante per $(x,v)$ al tempo $t$.
\end{definition}

\begin{definition}
    La soluzione del sistema newtoniano corrispondente dal dato iniziale $(x_0, v_0)$ in $\mathbb{R}^{2d}$ si indica con 
    $\Phi_N^t(x_0,v_0)$ e si chiama \textit{flusso del campo vettoriale newtoniano}
    \begin{equation}
        \Phi_N^t; \mathbb{R}^{2d}\rightarrow\mathbb{R}^{2d} ; \qquad \Phi_N^0 (x_0,v_0)= (x_0,v_0) 
    \end{equation}
\end{definition}

%esercizio 1  iterativo

% Reazione di radiazione di Lorentz
\subsection{Modelli di forza}
\subsubsection{Spazialmente uniformi}
In cui la forza non dipende dalla posizione: $f = f(v,t)$.
\begin{example}
    $f = -\gamma v $ Attrito viscoso
\end{example}
\begin{example}
    $f = q(E_0+ \frac{v}{c}\times B_0)$ Forza di Lorents in campi uniformi
\end{example}
\subsubsection{Puramente posizionali}
Forze indipendenti dalla velocità: $f= f(x,t)$. Distinguiamo inoltre le \textit{forze potenziali} in cui esiste un potenziale scalare $U(x,t)$
tale che:
\begin{equation}
    f= -\nabla_xU(x,t)= -\pdv{U}{x}(x,t)
\end{equation}
Le forze potenziali indipendenti dal tempo sono dette \textit{forze conservative}.
\begin{theorem}
    Per un sistema newtoniano conservativo, la funzione energia totale
    \begin{equation}
        H(x,\dot{x}):= K +U := \frac{1}{2}m\abs{\dot{x}}^2+U(x)
    \end{equation}
    è costante lungo le soluzioni della equazione del moto.
\end{theorem}
\begin{proof}
    Per una forza conservativa: $m\ddot{x}= -\nabla_xU(x)$
    \begin{equation*}
        \implies \dot{x}\cdot (m\ddot{x})= -\dot{x} \cdot\nabla_xU \implies m\dv{}{t}\left( \frac{\abs{\dot{x}}^2}{2} \right)= -\dv{}{t}U(x(t))
    \end{equation*}
    \begin{equation}
        \implies \dv{}{t}\left( \frac{m\abs{\dot{x}}^2}{2} +U(x) \right) = 0
    \end{equation}
\end{proof}

Il moto avviene nell'insieme di livello $H^{-1}(E)$ dove $E = H(x_0,v_0)$, detta \textit{superficie equipotenziale}.
\begin{equation}
    \Sigma_E:= H^{-1}(E)=\left\{ (x,v) \in \mathbb{R}^{2d}:H(x,v)=E  \right\} 
\end{equation}

\begin{proposition}
    Il moto è sempre tangente alla supericie equipotenziale.
\end{proposition}
\begin{proof}
    \begin{equation*}
        H(x,v)= m\frac{\abs{v}^2}{2}+U(x) \implies 
        \nabla H(x,v)= 
        \begin{pmatrix}
            \nabla_xU\\mv 
        \end{pmatrix}
        \qquad \dot{X}= 
        \begin{pmatrix}
            v\\\frac{f}{m}
        \end{pmatrix}
    \end{equation*}
    \begin{equation*}
        \nabla H \cdot \dot{X}=  
        \begin{pmatrix}
            \nabla_xU\\mv 
        \end{pmatrix} \cdot
        \begin{pmatrix}
            v\\\frac{f}{m}
        \end{pmatrix}
        = v\cdot\nabla_xU +v\cdot f = -v \cdot f +v\cdot f = 0 
    \end{equation*}
\end{proof}

\begin{remark}
    $\Sigma_E $ ha dimensione $2d-1$
\end{remark}

Suppongo di avere una forza genererica per cui $m\ddot{x}= f $, cerchiamo U(x) tale che $H(x,v)$ si conservi lungo i moti.
\begin{equation*}
    \dv{}{t}H(x(t),v(t))= \nabla_xH\cdot\dot{x}+\nabla_v\cdot\dot{v}= \nabla_xU\cdot + mv\cdot \frac{f}{m}\equiv0
\end{equation*}
\begin{equation}
    \implies v \cdot \left( f + \nabla_xU \right)=0
\end{equation}
Questo implica che possono esserci componenti della forza perpendicolari alla velocità dette \textit{giroscopiche}.
\begin{equation}
    f= -\nabla_xU + v\times (\dots)
\end{equation}

\subsection{Forze centrali}
