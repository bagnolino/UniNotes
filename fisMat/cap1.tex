\subsection{Meccanica Newtoniana}
\begin{equation}
    f = m a 
\end{equation}
Il secondo principio della dinamica di Newton è l'equazione del moto di una particella di massa $m$ 
soggetta a una forza $f$, $a$ è l'accelerazione.\\ 
\begin{equation}
    m \ddot{x} = f( x(t),\dot{x}(t), t) \qquad \text{ ove } \quad v = \dot{x}(t) = \dv{x}{t}\quad a = \ddot{x}(t)= \dv[2]{x}{t}
\end{equation}
La forza $f$ è una funzione assegnata del tipo $f: \mathbb{R}^d\times \mathbb{R}^d \times \mathbb{R} \rightarrow \mathbb{R}^d$\\
L'equazione di newton è un'equazione differenziale ordinaria ODE di secondo ordine con incognita $x(t)$. 
In base ai dati iniziali e alla funzione $f$ possono valere o meno i teoremi di esistenza e unicità $\exists!$ locale e globale.\\ 
$x(t)$ è una curva del tipo $I \subseteq \mathbb{R}\quad x: I \rightarrow \mathbb{R}^d $. L'equazione può essere 
riscritta nella forma di sistema equivalente:
\begin{equation}
    \begin{cases}
        \dot{x}(t) = v(t)\\ 
        \dot{v}(t) =\frac{1}{m} f( x(t), v(t), t)
    \end{cases}
    \implies
    \qquad
    \dot{X}(t)= F(t)
\end{equation}
ove:
\begin{equation*}
    X (t) = \begin{pmatrix}
        x (t) \\ 
        v (t)
    \end{pmatrix} \subseteq \mathbb{R}^d \times \mathbb{R}^d \simeq \mathbb{R}^{2d} \qquad
    F (t) = \begin{pmatrix}
        v(t) \\
        f(x(t), v(t),t)
    \end{pmatrix}
\end{equation*}
