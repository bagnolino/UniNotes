\section{Meccanica Newtoniana}

\subsection{L'equazione di Newton}
\begin{equation}
    f = m a 
\end{equation}
Il \textit{secondo principio della dinamica} di Newton è l'equazione del moto di una particella di massa $m$ 
soggetta a una forza $f$, $a$ è l'accelerazione.\\ 
\begin{equation}
    m \ddot{x} = f( x(t),\dot{x}(t), t) \qquad \text{ ove } \quad v = \dot{x}(t) = \dv{x}{t}\quad a = \ddot{x}(t)= \dv[2]{x}{t}
\end{equation}

La forza $f$ è una funzione assegnata del tipo $f: \mathbb{R}^d\times \mathbb{R}^d \times \mathbb{R} \rightarrow \mathbb{R}^d$\\
L'equazione di newton è un'\textit{equazione differenziale ordinaria} ODE di secondo ordine con incognita $x(t)$. 
In base ai dati iniziali e alla funzione $f$ possono valere o meno i teoremi di esistenza e unicità $\exists!$ locale e globale.\\ 
$x(t)$ è una curva del tipo $I \subseteq \mathbb{R}\quad x: I \rightarrow \mathbb{R}^d $. L'equazione può essere 
riscritta nella forma di sistema equivalente:
\begin{equation}
    \begin{cases}
        \dot{x}(t) = v(t)\\ 
        \dot{v}(t) =\frac{1}{m} f( x(t), v(t), t)
    \end{cases}
    \implies
    \qquad
    \dot{X}(t)= F(t)
\end{equation}
ove:
\begin{equation*}
    X (t) = \begin{pmatrix}
        x (t) \\ 
        v (t)
    \end{pmatrix} \subseteq \mathbb{R}^d \times \mathbb{R}^d \simeq \mathbb{R}^{2d} \qquad
    F (t) = \begin{pmatrix}
        v(t) \\
        f(x(t), v(t),t)
    \end{pmatrix}
\end{equation*}

Il moto può essere univocamente determinato se assegno il dato iniziale $x(t_0)=x_0,\;\dot{x}(t_0)= v(t_0)=v_0$.\\
Conoscendo i dati iniziali, cerchiamo una soluzione locale tramite sviluppo di Taylor: 
\begin{equation*}
    x(t_0) \text{ noto}\implies x(t_0+h)= x(t_0)+\dot{x}(t_0) h +\frac{1}{2!}\ddot{x}(t_0) h^2 +\frac{1}{3!}\dddot{x}(t_0)h^3  + o(h^4)= 
\end{equation*}
\begin{equation*}
   = x_0 +v_0 h + \frac{f}{2m}(x_0,v_0,t_0) h^2 + \dots
\end{equation*}

Si può scrivere:
\begin{equation}
    \begin{cases}
        \dd{x}= v(t)\dd{t}\\
        \dd{v}= \frac{1}{m}f(x(t),v(t),t)\dd{t} 
    \end{cases}
\end{equation}
\begin{equation*}
    (x,v) \text{ in } t \implies  \text{ in } t +\dd{t} \quad (x',v')= (x+\dd{x},v+\dd{v})= (x+v\dd{t},v+ \frac{f}{m}\dd{t})
\end{equation*}

\begin{definition}
    \begin{equation*}
        \begin{pmatrix}
            x\\v  
        \end{pmatrix}\in \mathbb{R}^{2d} 
    \end{equation*}
    è detto \textit{spazio delle fasi} del sistema. 
\end{definition}

\begin{definition}
    La curva soluzione $t\rightarrow(x(t),v(t))$ si chiama \textit{curva di fase}, distinta da $t\rightarrow x(t)$ che si chiama 
    \textit{traiettoria o orbita}.
\end{definition}

\begin{definition}
    \begin{equation*}
        F=\begin{pmatrix}
            v\\\frac{1}{m}f 
        \end{pmatrix}
    \end{equation*}
    si chiama \textit{campo vettoriale newtoniano} ed è geometricamente il vettore tangente alla curva di fase passante per $(x,v)$ al tempo $t$.
\end{definition}

\begin{definition}
    La soluzione del sistema newtoniano corrispondente dal dato iniziale $(x_0, v_0)$ in $\mathbb{R}^{2d}$ si indica con 
    $\Phi_N^t(x_0,v_0)$ e si chiama \textit{flusso del campo vettoriale newtoniano}
    \begin{equation}
        \Phi_N^t; \mathbb{R}^{2d}\rightarrow\mathbb{R}^{2d} ; \qquad \Phi_N^0 (x_0,v_0)= (x_0,v_0) 
    \end{equation}
\end{definition}

%esercizio 1  iterativo

% Reazione di radiazione di Lorentz



\subsection{Modelli di forza}

\subsubsection{Spazialmente uniformi}
In cui la forza non dipende dalla posizione: $f = f(v,t)$.
\begin{example}
    $f = -\gamma v $ Attrito viscoso
\end{example}
\begin{example}
    $f = q(E_0+ \frac{v}{c}\times B_0)$ Forza di Lorents in campi uniformi
\end{example}


\subsubsection{Puramente posizionali}
Forze indipendenti dalla velocità: $f= f(x,t)$. Distinguiamo inoltre le \textit{forze potenziali} in cui esiste un potenziale scalare $U(x,t)$
tale che:
\begin{equation}
    f= -\nabla_xU(x,t)= -\pdv{U}{x}(x,t)
\end{equation}

Le forze potenziali indipendenti dal tempo sono dette \textit{forze conservative}.
\begin{theorem}
    Per un sistema newtoniano conservativo, la funzione energia totale
    \begin{equation}
        H(x,\dot{x}):= K +U := \frac{1}{2}m\abs{\dot{x}}^2+U(x)
    \end{equation}
    è costante lungo le soluzioni della equazione del moto.
\end{theorem}
\begin{proof}
    Per una forza conservativa: $m\ddot{x}= -\nabla_xU(x)$
    \begin{equation*}
        \implies \dot{x}\cdot (m\ddot{x})= -\dot{x} \cdot\nabla_xU \implies m\dv{}{t}\left( \frac{\abs{\dot{x}}^2}{2} \right)= -\dv{}{t}U(x(t))
    \end{equation*}
    \begin{equation}
        \implies \dv{}{t}\left( \frac{m\abs{\dot{x}}^2}{2} +U(x) \right) = 0
    \end{equation}
\end{proof}

Il moto avviene nell'insieme di livello $H^{-1}(E)$ dove $E = H(x_0,v_0)$, detta \textit{superficie equipotenziale}.
\begin{equation}
    \Sigma_E:= H^{-1}(E)=\left\{ (x,v) \in \mathbb{R}^{2d}:H(x,v)=E  \right\} 
\end{equation}

\begin{proposition}
    Il moto è sempre tangente alla supericie equipotenziale.
\end{proposition}
\begin{proof}
    \begin{equation*}
        H(x,v)= m\frac{\abs{v}^2}{2}+U(x) \implies 
        \nabla H(x,v)= 
        \begin{pmatrix}
            \nabla_xU\\mv 
        \end{pmatrix}
        \qquad \dot{X}= 
        \begin{pmatrix}
            v\\\frac{f}{m}
        \end{pmatrix}
    \end{equation*}
    \begin{equation*}
        \nabla H \cdot \dot{X}=  
        \begin{pmatrix}
            \nabla_xU\\mv 
        \end{pmatrix} \cdot
        \begin{pmatrix}
            v\\\frac{f}{m}
        \end{pmatrix}
        = v\cdot\nabla_xU +v\cdot f = -v \cdot f +v\cdot f = 0 
    \end{equation*}
\end{proof}

\begin{remark}
    $\Sigma_E $ ha dimensione $2d-1$
\end{remark}

Suppongo di avere una forza genererica per cui $m\ddot{x}= f $, cerchiamo U(x) tale che $H(x,v)$ si conservi lungo i moti.
\begin{equation*}
    \dv{}{t}H(x(t),v(t))= \nabla_xH\cdot\dot{x}+\nabla_v\cdot\dot{v}= \nabla_xU\cdot + mv\cdot \frac{f}{m}\equiv0
\end{equation*}
\begin{equation}
    \implies v \cdot \left( f + \nabla_xU \right)=0
\end{equation}
Questo implica che possono esserci componenti della forza perpendicolari alla velocità dette \textit{giroscopiche}.
\begin{equation}
    f= -\nabla_xU + v\times (\dots)
\end{equation}


\subsubsection{Forze centrali}

\begin{definition}
    Una forza $f(x,v,t)$ si dice \textit{centrale} se è diretta come il vettore posizione $x$ a ogni istante, ossia:
    \begin{equation}
        f(x,v,t)= \Psi(x,v,t)x= \varphi(x,v,t)\hat{x}
    \end{equation}
    dove $\Psi,\varphi: \mathbb{R}^d\times \mathbb{R}^d \times \mathbb{R} \rightarrow \mathbb{R}$ e $\varphi=\Psi\abs{x}, \;\hat{x}=\frac{x}{\abs{x}}$.
\end{definition}

\begin{definition}
    L'origine, in un campo di forze centrali, si chiama \textit{centro della forza}. $\varphi$ è lo \textit{scalare della forza}.
\end{definition}

\begin{proposition}
    La quantità vettoriale $\ell(x,\dot{x}):= x\times m\dot{x}$, detta \textit{momento angolare}, si conserva lungo i moti newtoniani con forze centrali.
\end{proposition}
\begin{proof}
    \begin{equation*}
        m\ddot{x}= f= \varphi\hat{x};  \qquad \dv{\ell}{t}= \dv{t}\left( x(t)\times m\dot{x}(t)\right)= 
        \dot{x}(t)\times m\dot{x}(t)+ x(t)\times m\ddot{x}(t)=
    \end{equation*}
    \begin{equation}
        m(\dot{x}\times\dot{x})+x\times f= \varphi x \times \hat{x}=0
    \end{equation}
\end{proof}

\begin{proposition}
    Il moto della particella dunque si svolge sul piano ortogonale a $ell$ di equazione: $\ell\cdot x =0$
\end{proposition}
\begin{proof}
    Dimostro che il piano $\pi:\ell\cdot x=0$ è invariante:
    \begin{equation}
        \ell\cdot x= (x(t)\times m\dot{x}(t))\cdot x(t)=0 \qquad\forall \, t
    \end{equation}
\end{proof}

\begin{remark}
    Dunque i moti centrali sono piano e si può ridurre di una dimensione
\end{remark}
\begin{definition}
    La \textit{velocità areolare } è l'area spazzata dal raggio vettore $x(t)$ nell'unità di tempo.
\end{definition}

\begin{proposition}
    La velocità areolare nei moti centrali è costante. Questa è anche detta \textit{seconda legge di Keplero}.
\end{proposition}
\begin{proof}
    Considero la variazione di area $\Delta A$ spazzata dal raggio vettore $x(t)$ in un intervallo di tempo $\Delta t$:
    \begin{equation*}
        \Delta A \simeq \frac{1}{2} \abs{ x(t) \times (x(t+\Delta t) - x(t)) } = \frac{1}{2} \abs{ x(t) \times \dot{x}(t) } \Delta t
    \end{equation*}
    \begin{equation}
        \dot{A}(t) = \lim_{\Delta t \to 0} \frac{\Delta A}{\Delta t} = \frac{1}{2} \abs{ x(t) \times \dot{x}(t) }
    \end{equation}
    \begin{equation}
        \dot{A}(t) = \frac{1}{2m} \abs{ x(t) \times m\dot{x}(t) } = \frac{1}{2m} \abs{ \ell } \qquad \textit{Legge della vel. areolare}
    \end{equation}
    Poiché $\ell$ si conserva nei moti centrali, segue che $\dot{A}(t)$ è costante.
\end{proof}

\subsubsection{Forze centrali autonome, posizionali a simmetria sferica}
Forze in cui la forza dipende solamente dal modulo del vettore posizione:
\begin{equation}
    f:= \varphi(\abs{x})\hat{x}
\end{equation}

Se $x= Rx'$ dove $R$ è una matrice di rotazione:
\begin{equation}
    \abs{x}= \abs{Rx'}= \sqrt{(Rx')\cdot(Rx')}= \sqrt{x'\cdot(R^\intercal R)x'}= \abs{x'}
\end{equation}
\begin{equation}
    \hat{x}=\frac{x}{\abs{x}}= \frac{Rx'}{\abs{Rx'}}= \frac{Rx'}{\abs{x'}}=R\hat{x}'
\end{equation}

\begin{proposition}
    $f= \varphi(\abs{x})\hat{x}$ è conservativa.    
\end{proposition}
\begin{proof}
    Consideriamo un potenziale:
    \begin{equation}
        U(\abs{x}):=-\int^\abs{x}\varphi(r)\dd{r}
    \end{equation}
    \begin{equation*}
        -\nabla_x U(\abs{x}) = -U'(\abs{x}) \nabla_x (\abs{x}) = -U'(\abs{x}) \nabla_x (\sqrt{x \cdot x}) 
    \end{equation*}
    \begin{equation*}
        = -U'(\abs{x}) \frac{\nabla_x (x \cdot x)}{2 \sqrt{x \cdot x}} = -U'(\abs{x}) \frac{2x}{2 \abs{x}} = -U'(\abs{x}) \hat{x}
\end{equation*}
\end{proof}

\begin{remark}
    Si ricava che la forma più generale per cui una forza centrale $f= \varphi(x)\hat{x}$ sia conservativa è $\varphi(x)=\varphi(\abs{x})$.
\end{remark}
\begin{proof}
    Usando le coordinate polari sferiche:     ($r = \abs{x}$)
    \begin{equation}
        \begin{cases}
            x_1 = r \sin\theta \cos\phi \\
            x_2 = r \sin\theta \sin\phi \\
            x_3 = r \cos\theta 
        \end{cases}
    \end{equation}
    Calcoliamo le derivate parziali di $U$:
    \begin{equation*}
        \pdv{U}{\theta}= \pdv{U}{x_1}\pdv{x_1}{\theta}+\pdv{U}{x_2}\pdv{x_2}{\theta}+\pdv{U}{x_3}\pdv{x_3}{\theta}=
         \left( \nabla_xU \right)\cdot\pdv{x}{\theta}= -\frac{\varphi(x)}{\abs{x}}\frac{1}{2}\pdv{\theta}\left( x\cdot x \right)
    \end{equation*}
    \begin{equation}
        =-\varphi(x)\frac{1}{\abs{x}}x\cdot \pdv{x}{\theta}= - \varphi(x)\frac{1}{2\abs{x}}\pdv{\theta}\abs{x}^2
        =- \varphi(x)\frac{1}{2\abs{x}}\pdv{r^2}{\theta}\equiv0 
    \end{equation}
    Perché $r$ e $\theta$ sono linearmente indpendenti.\\
    Analogamente si dimostra che $\pdv{U}{\phi}=0$.
    \begin{equation*}
        U = U(r) = U(\abs{x})\implies \nabla_xU= U'(\abs{x})\hat{x}
    \end{equation*}
    \begin{equation*}
        \varphi(x)\hat{x}=-U'(\abs{x})\hat{x}\implies \varphi(x)= -U'(\abs{x})
    \end{equation*}
\end{proof}

Le forza fondamentali di questo tipo sono la \textit{forza gravitazionale} e la \textit{forza elettrostatica} 
con rispettivamente la \textit{legge di Hooke-Newton} e la \textit{legge di Coulomb}:
\begin{equation}
    \varphi_G(\abs{x})= -G\frac{Mm}{\abs{x}^2}\qquad \quad \varphi_{es}(\abs{x})=\frac{Qq}{\abs{x}^2}
\end{equation}

Un'altra forza importante è la \textit{forza elastica}, descritta dalla \textit{legge di Hooke}:
\begin{equation}
    \varphi_{el}(\abs{x})= -k\abs{x}
\end{equation}

%Accenno di Lagrange e Hamilton

\subsection{Problemi}

\subsubsection{Oscillatore armonico tridimensionale}
\begin{equation}
    m\ddot{x}= f \qquad f = \varphi(\abs{x})\hat{x}\qquad \varphi = -kr
\end{equation}
E' una forza centrale: $\ell= x\times m\dot{x}$ è costante.\\
E' uno dei tipi di forze che produce orbite elittiche. 
E' stato candidato per la forza gravitazionale, ma la forza non è centrata nei fuochi
e il periodo non dipende dall'ampiezza.\\

Per $\ell  \neq 0$, cambio le coordinate a un sistema in cui:
\begin{equation}
    x_3 \parallel \ell; \quad x_1,x_2 \in x\cdot\ell=0  \qquad (3D\rightarrow 2D)
\end{equation}

Risolviamo l'equazione del moto:
\begin{equation*}
    m\ddot{x}= -kx \implies \ddot{x}= -\omega^2 x \text{ con } \omega= \sqrt{\frac{k}{m}}
\end{equation*}
In componenti:
\begin{equation}
    \begin{cases}
        \ddot{x}_1= -\omega^2 x_1\\
        \ddot{x}_2= -\omega^2 x_2
    \end{cases}\implies
    x_1 = a \cos(\omega t +\phi)\quad x_2= b\cos(\omega t + \psi)
\end{equation}
\begin{remark}
    Il periodo $T$ non dipende da $a$ e $b$:
    \begin{equation}
        T = \frac{2\pi}{\omega}= 2\pi\sqrt{\frac{m}{k}}
    \end{equation} 
\end{remark}

Definiamo: $\theta:=\omega t + \phi, \quad\delta:=\psi-\phi$.
\begin{equation}
    \implies \begin{cases}
        x_1 = a\cos(\theta)\\
        x_2 = b\cos(\theta+\delta)
    \end{cases}
\end{equation}

Se $\delta=0\implies x_1= a\cos(\theta),\;x_2=b\cos(\theta)\implies x_1=\frac{a}{b}x_2$ è il caso lineare in cui $\ell = 0$.
Se $\delta= \pi$ è uguale al caso precedente.\\
Se $\delta\neq 0$:
\begin{equation*}
    \begin{cases}
        x_1 = a\cos(\theta)\\
        x_2 = b(\cos(\theta)\cos(\delta)-\sin(\theta)\sin(\delta))
    \end{cases}
\end{equation*}
\begin{equation}
    \cos(\theta)=\frac{x_1}{a} \quad \frac{x_2}{b}= \frac{x_1}{a}\cos(\delta)-\sin(\theta)\sin(\delta)\implies
    \sin(\theta)= \frac{1}{\sin(\delta)}\left(  \frac{x_1}{a}\cos(\delta)-\frac{x_2}{b}\right)
\end{equation}

Applico l'ugualianza:
\begin{equation*}
    \frac{x_1^2}{a^2} + \frac{1}{\sin^2\delta}
      \left( \frac{x_1}{a}\cos\delta - \frac{x_2}{b} \right)^2= 1
\end{equation*}
\begin{equation*}
    \frac{x_1^2}{a^2} + \frac{1}{\sin^2\delta}
    \left( \frac{x_1^2}{a^2}\cos^2\delta - 2\frac{x_1 x_2}{a b}\cos\delta + \frac{x_2^2}{b^2} \right) = 1
\end{equation*}
\begin{equation}
    \left( 1 + \frac{\cos^2\delta}{\sin^2\delta} \right) \frac{x_1^2}{a^2}
    + \frac{1}{\sin^2\delta}\frac{x_2^2}{b^2}- 2\frac{\cos\delta}{\sin^2\delta}\frac{x_1 x_2}{a b} = 1
\end{equation}
Riscriviamola nella forma $x\cdot Mx = 1$:
\begin{equation}
    M = 
    \begin{pmatrix}
        \dfrac{1}{a^2 \sin^2 \delta} & -\dfrac{\cos \delta}{a b \sin^2 \delta} \\[1.2ex]
        -\dfrac{\cos \delta}{a b \sin^2 \delta} & \dfrac{1}{b^2 \sin^2 \delta}
    \end{pmatrix}
\end{equation}
$M$ è matrice simmetrica e reale, dunque si può applicare il teorema spettrale:
\begin{equation}
    \implies \exists\,R, \;R^\intercal R = 1 \quad t.c. \quad R^\intercal MR = \operatorname{diag} (m_1,m_2)
\end{equation}
Dove $m_1$ e $m_2$ sono gli autovalori reali di $M$. Esiste dunque un vettore $y:= R^\intercal x$ per cui:
\begin{equation}
    x\cdot Mx= (Ry)\cdot M(Ry)= y\cdot(R^\intercal MR)y= y\cdot \left( \operatorname{diag} (m_1,m_2) y\right)= m_1 y_1^2+m_2y_2^2=1
\end{equation}
Che è l'equazione per un elisse.
\begin{remark}
    Possiamo riscriverla come: $m_1 y_1^2+m_2y_2^2=\frac{y_1^2}{A^2}+\frac{y_2^2}{B^2}= 1$
\end{remark}
\begin{proof}
    \begin{equation*}
       \tr(M )>0, \det(M ) = \frac{1}{a^2b^2\sin[2](\delta)}>0\implies m_1+m_2>0, \quad m_1m_2>0 
    \end{equation*}
\end{proof}

\paragraph{RECAP Elisse}
La definizione geometetrica di un elisse è il luogo dei punti tale che $d_1+d_2= 2a$.
La definizione cartesiana è:
\begin{equation}
    \frac{x_1^2}{a^2}+ \frac{x_2^2}{b^2}=1 \quad \text{ con } a,b \in \mathbb{R}
\end{equation}
La definizione in coordinate polari:
\begin{equation}
    r = \frac{P}{1+\varepsilon\cos(\theta)} \quad \text{ con } P,\varepsilon \in \mathbb{R}
\end{equation}
Nella definizione in coordinate polari il centro è uno dei due fuochi, $\varepsilon \in [0,1)$ è detta eccentricità
e il parametro $P=r(\frac{\pi}{2})$.
Dimostriamo che queste tre definizioni siano equivalenti.
\begin{proof}
    1) $\leftrightarrow$ 2)\\
    Consideriamo le distanze di un punto generico $(x, y)$ dai due fuochi $F_1, F_2$. Sappiamo che:
    \begin{equation}
        \begin{cases}
            d_1^2 = (x - f)^2 + y^2 \\
            d_2^2 = (x + f)^2 + y^2
        \end{cases}
        \qquad
        (d_1 + d_2)^2 = 4a^2
    \end{equation}
    \begin{equation}
        d_1^2 + d_2^2 + 2 d_1 d_2 = 4a^2 \implies 4 d_1^2 d_2^2 = \left[ 4a^2 - (d_1^2 + d_2^2) \right]^2
    \end{equation}
    Sviluppando i conti si ottiene:
    \begin{equation}
        (a^2 - f^2)x^2 + a^2 y^2 = a^2(a^2 - f^2) \implies \frac{x^2}{a^2} + \frac{y^2}{a^2 - f^2} = 1
    \end{equation}

    2) $\leftrightarrow$ 3)\\
    Usando il cambio di coordinate polari:
    \begin{equation}
        \begin{cases}
            x = f + r \cos\theta \\
            y = r \sin\theta
        \end{cases}
    \end{equation}
    Sostituendo nell'equazione cartesiana:
    \begin{equation*}
        (a^2 - f^2)(f + r\cos\theta)^2 + a^2 r^2 \sin^2\theta = a^2(a^2 - f^2)
    \end{equation*}
    \begin{equation*}
        (a^2 - f^2)f^2 + 2f(a^2 - f^2)r\cos\theta + \left[(a^2 - f^2)\cos^2\theta + a^2\sin^2\theta\right] r^2 = a^2(a^2 - f^2)
    \end{equation*}
    \begin{equation*}
        \left[(a^2 - f^2)\cos^2\theta + a^2\sin^2\theta\right] r^2 + 2f(a^2 - f^2) r\cos\theta + (a^2 - f^2)f^2 - a^2(a^2 - f^2) = 0
    \end{equation*}
    \begin{equation}
        \left(a^2 - f^2\cos^2\theta\right) r^2 + 2f(a^2 - f^2) r\cos\theta - (a^2 - f^2)^2 = 0
    \end{equation}
    Risolvendo come equazione di secondo grado in $r$:
    \begin{equation*}
        r = \frac{-(a^2 - f^2)f\cos\theta + \sqrt{(a^2 - f^2)^2 a^2}}{a^2 - f^2\cos^2\theta}
    \end{equation*}
    Considerando solo la soluzione positiva:
    \begin{equation}
        r = \frac{a^2 - f^2}{a + f\cos\theta}=\frac{b^2}{a + f\cos\theta}
        = \frac{P}{1 + \varepsilon\cos\theta} \quad \text{con } P:= \frac{b^2}{a}, \varepsilon:= \frac{f}{a}
    \end{equation}
\end{proof}
\begin{remark}
    Per $\varepsilon \notin [0,1)$, ottengo le altre sezioni coniche: parabola ($\varepsilon=1$) e iperbole ($\varepsilon>1$).
\end{remark}


\subsubsection{Problema di Keplero}
Si studi il moto del corpo soggetto alla forza gravitazionale:
\begin{equation}
    m\ddot{x}=-\frac{k}{\abs{x}^2}\hat{x} \quad \text{ con } k>0
\end{equation}

E' una forza centrale quindi: $\dot{\ell}=0$ e ho un moto piano in $\ell\cdot x=0$.
\begin{equation}
    \begin{cases}
        m\ddot{x}_1= -k\frac{x_1}{\abs{x}^3}\\
        m\ddot{x}_2= -k\frac{x_2}{\abs{x}^3}
    \end{cases}
\end{equation}

E' opportuno considerare il problema in coordinate polari piane: $x_1= r\cos(\theta), \quad x_2=r\sin(\theta)$. 
\begin{equation}
    \pdv{x}{r}= \pdv{}{r}\begin{pmatrix}
        x_1\\x_2
    \end{pmatrix}=\begin{pmatrix}
        \cos(\theta)\\\sin(\theta)
    \end{pmatrix}=:\hat{e}_r 
\end{equation}
\begin{equation}
    \pdv{x}{\theta}= r \begin{pmatrix}
        -\sin(\theta)\\\cos(\theta)
    \end{pmatrix}\implies 
    \hat{e}_\theta:=\frac{1}{r}\pdv{x}{\theta}= \begin{pmatrix}
        -\sin(\theta)\\\cos(\theta)
    \end{pmatrix}
\end{equation}
\begin{equation}
    \dv{\hat{e}_r(\theta)}{\theta} = \hat{e}_\theta(\theta) ; \qquad
    \dv{\hat{e}_\theta(\theta)}{\theta} = -\hat{e}_r(\theta) ; \qquad
    \dv{\hat{e}_r(\theta)}{r} = 0 ; \qquad
    \dv{\hat{e}_\theta(\theta)}{r} = 0\quad
\end{equation}

Chiamiamo semplicemente $\hat{e}_r= \hat{r},\;\hat{e}_\theta=\hat{\theta}$.
\begin{equation}
    x = x_1\hat{e}_1+x_2\hat{e}_2= r\hat{r} \quad 
    \dot{x}= \dot{r}\hat{r} + r\dv{\hat{r}}{t}=  \dot{r}\hat{r} + r \dot{\theta}\hat{\theta}
\end{equation}
\begin{equation}
    \ddot{x}=\ddot{r}\hat{r}+\dot{r}\dot{\theta}\hat{\theta} + \dot{r}\dot{\theta}\hat{\theta}  + r\ddot{\theta}\hat{\theta}-r(\dot{\theta})^2\hat{r}
    =\begin{pmatrix}
        \ddot{r}-r\dot{\theta}^2\\
        r\ddot{\theta}+2\dot{r}\dot{\theta}
    \end{pmatrix}
\end{equation}

Quindi l'equazione del moto in coordinate polari diventa:
\begin{equation}
    m\left[(\ddot{r}-r\dot{\theta}^2)\hat{r}+(2\dot{r}\dot{\theta}+r\ddot{\theta})\hat{\theta}\right]= -\frac{k}{r^2}\hat{r}
    \quad\leftrightarrow \quad
    \begin{cases}
        m(\ddot{r}-r\dot{\theta}^2)=-\frac{k}{r^2}\\
        2\dot{r}\dot{\theta}+r\ddot{\theta}=0
    \end{cases}
\end{equation}

\begin{remark}
    L'equazione lungo $\hat{\theta}$ corrisponde alla legge di conservazione di $\ell$.
\end{remark}
\begin{proof}
    Moltiplicando per $r$:
    \begin{equation*}
        2r\dot{r}\dot{\theta}m + r^2\ddot{\theta}= m \dv{t}(r^2)\dot{\theta}+mr^2\dv{t}(\dot{\theta})
    \end{equation*}
    \begin{equation}
        \implies \dv{t}\left( mr^2\dot{\theta} \right)=0\implies mr^2\dot{\theta}= \text{ costante}
    \end{equation}
    \begin{equation}
        \ell= x \times m\dot{x}= r\hat{r}\times m \left( \dot{r}\hat{r}+r\dot{\theta} \hat{\theta}\right)= mr^2\dot{\theta}\hat{z}   
    \end{equation}
\end{proof}
Possiamo riscrivere le nostre equazioni come:
\begin{equation}
    \begin{cases}
        m\ddot{r}= mr\dot{\theta^2}-\frac{k}{r^2}= \frac{\ell^2}{mr^3}-\frac{k}{r^2}\\
        \dot(\theta)= \frac{\ell}{mr^2}
    \end{cases}
\end{equation}

Cerchiamo ora la forma dell'orbita: $\theta\rightarrow r(\theta)$. Senza perdità di generalità abbiamo $l>0$, 
quindi esiste una funzione $t\rightarrow\theta(t)$ monotona e crescente e dunque esiste anche la sua inversa $\theta\rightarrow t(\theta)$.
Per cui $\exists \,r(t(\theta))= r(\theta)$. 
\begin{equation}
    \dv{r}{\theta}=\dv{r}{t}\dv{t}{\theta}= \frac{\dot{r}}{\dot{\theta}}\implies \dot{r}= \dot{\theta}\dv{r}{\theta}= \frac{\ell}{mr^2}\dv{r}{\theta}
\end{equation}
Potrei calcolare anche $\ddot{r}$ e risolvere l'equazione differenziale, ma è più comodo lavorare considerando l'equazione di conservazione dell'energia:
\begin{equation*}
    E= \frac{m}{2}\abs{\dot{x}}^2 + U(\abs{x})= \frac{m}{2}\abs{\dot{r}\hat{r}+ r\dot{\theta}\hat{\theta}}^2-\frac{k}{r}=
\end{equation*}
\begin{equation*}
    \frac{m}{2}\left( \dot{r}^2 + r^2\dot{\theta}^2  \right)-\frac{k}{r}= \frac{m}{2}\left( \dot{r}^2 + \frac{\ell^2}{m^2r^2}\right)-\frac{k}{r}=
\end{equation*}
\begin{equation*}
    \frac{m}{2}\left( \frac{\ell^2}{mr^2}\dv{r}{\theta} \right)^2+ \frac{\ell^2}{2mr^2}-\frac{k}{r}= 
    \frac{\ell^2}{2mr^4}\left( \dv{r}{\theta} \right)^2+\frac{\ell^2}{2mr^2}-\frac{k}{r}
\end{equation*}
Introduciamo il cambio di variabile:
\begin{equation}
    u(\theta):= \frac{1}{r(\theta)}\implies \dv{r}{\theta}= -\frac{1}{u^2}\dv{u}{\theta}
\end{equation}
\begin{equation}
    E=\frac{\ell^2u^4}{2m}\left( +\frac{1}{u^4} \right)\left( \dv{u}{\theta} \right)^2+\frac{\ell^2}{2m}u^2-ku=
    \frac{\ell^2}{2m}\left( \dv{u}{\theta} \right)^2+\frac{\ell^2}{2m}u^2-ku
\end{equation}
Deriviamo il tutto rispetto a $\theta$:
\begin{equation}
    0 = \frac{\ell^2} 2\dv{u}{\theta}\dv[2]{u}{\theta}+\frac{\ell^2}{2m}2u\dv{u}{\theta} -k\dv{u}{\theta}
\end{equation}
Considerando $\dv{u}{\theta}\neq 0$ in ogni punto, otteniamo l'\textit{equazione di Binet:}
\begin{equation}
    u''(\theta)+u(\theta)- \frac{km}{\ell^2}=0
\end{equation}
Che si risolve facilmente:
\begin{equation}
    \frac{1}{r(\theta)}=u(\theta)= \frac{km}{\ell^2}+A\cos(\theta-\theta_0)\implies r(\theta)= 
    \frac{\frac{\ell^2}{km}}{1+\frac{A\ell^2}{km}\cos(\theta)}= \frac{P}{1+\varepsilon\cos(\theta)}
\end{equation}






