\documentclass[12pt,a4paper]{article}
\usepackage[utf8]{inputenc}
\usepackage[italian]{babel}
\usepackage{amsmath,amssymb,amsthm}
\usepackage{physics}  
\usepackage{graphicx}
\usepackage{hyperref}
\usepackage{geometry}
\usepackage{tikz}



\geometry{margin=2.5cm}

\setlength{\parindent}{0pt}
%\setcounter{tocdepth}{2}

\usetikzlibrary{arrows.meta} 
\usetikzlibrary{patterns}


\theoremstyle{definition}
\newtheorem{theorem}{Teorema}
\newtheorem{definition}{Definizione}
\newtheorem{example}{Esempio}
\newtheorem{proposition}{Proposizione}
\newtheorem{lemma}{Lemma}
\newtheorem{remark}{Nota}

\title{Fisica 2 - Appunti}
\author{Davide Bagnoli}
\date{1 semestre 2024/2025}




\begin{document}
\maketitle
\tableofcontents
\newpage

\section{Analisi vettoriale}
\subsection{Note}
\begin{proposition}
    \begin{equation}
    \div(\curl \vb{A})= 0
    \end{equation}
\end{proposition}
\begin{proof}
$
\grad_{x}\vdot(\curl{\vb{A}})_{x}=\pdv{x}(\pdv{A_z}{y}-\pdv{A_y}{z})$
se $\vb{A}$ è $\mathcal{C}^{2}$ le derivate si scambiano e si elidono con gli altri termini.
\end{proof}

\begin{proposition}
    \begin{equation}
    \curl(\grad \Psi)=0
    \end{equation}
\end{proposition}

\begin{proof}
 $(\curl(\grad\Psi))_{x} = \vu{x}\left( \pdv{\Psi}{y}{z} -\pdv{\Psi}{z}{y}\right)=0$ se $\vb{A}$ è $\mathcal{C}^{2}$   
\end{proof}

\begin{theorem}
    $\forall$p $\div \vb{A}(p)= 0 \implies \exists\vb{C} $ campo vettoriale t.c. $\vb{A} = \curl \vb{C}$
\end{theorem}

\begin{theorem}
    $\forall$p $\curl \vb{A}(p)= 0 \implies \exists\Psi $ campo scalare t.c. $\vb{A} = \grad \Psi$
\end{theorem}

\newtheorem*{TS}{Teorema di Strokes}
\begin{TS}
    \begin{equation}
    \oint_{\Gamma}\vb{C}\cdot \dd{\vb{s}}= \int_{\Sigma}(\curl \vb{C})\cdot \vu{n} \dd{\Sigma}
    \end{equation}
\end{TS}

\begin{proof}
    Divido la superficie in quadrati abbastanza piccoli da considerarsi piani
    $\oint_{\Gamma}\vb{C}\cdot \dd{\vb{s}}= \sum_{i=1}\oint_{\Gamma_{i}}\vb{C}\cdot \dd{\vb{s}}$\\
    $\oint_{\Gamma_{i}}\vb{C}\cdot \dd{\vb{s}}= C_{x}(1)\delta x +C_{y}(2)\delta y -C_{x}(3)\delta x -C_{x}(4)\delta y $ dove $C_{x}$ e $C_{y}$ sono costanti.
    \\$[C_{x}(1)-C_{x}(3)]\delta x = [C_{x}(1)-C_{x}(1)-\pdv{C_x}{y}\delta y]\delta x \simeq -\pdv{C_x}{y}\delta y\delta x$
    \\$[C_{y}(2)-C_{y}(4)]\delta x = [C_{y}(2)-C_{y}(2)-\left( -\pdv{C_y}{x} \right)\delta x]\delta y \simeq +\pdv{C_y}{x}\delta x\delta y$
    \\$ \sum_{i=1}\oint_{\Gamma_{i}}\vb{C}\cdot \dd{\vb{s}}= \sum_{i=1}\left( +\pdv{C_y}{x} -\pdv{C_x}{y}\right) \delta y_{i}\delta x_{i}
    = \int_{\Sigma}\left( +\pdv{C_y}{x} -\pdv{C_x}{y}\right) \dd{\Sigma} =  \int_{\Sigma}\left( \curl \vb{C}\right) \dd{\Sigma}  $
\end{proof}

\newtheorem*{TG}{Teorema di Gauss}
\begin{TG}
    \begin{equation}
        \oint_{\Sigma} \vb{C}\cdot\vu{n}\dd{\Sigma}= \int_{V}\left( \div\vb{C} \right) \dd{\tau}
    \end{equation}
\end{TG}
\begin{proof}
    Posso dividere il volume della superficie in tanti piccoli cubetti
    \\ $\oint_{\Sigma} \vb{C}\cdot\vu{n}\dd{\Sigma}= \sum_{i=1} \oint_{\Sigma_{i}} \vb{C}\cdot\vu{n} \dd{\Sigma_{i}}$ 
    \\$\Phi(1) = -\int C_{y}\dd{x}\dd{z} =- C_{y}(1)\delta x \delta z$ \quad
    $\Phi(2) =  C_{y}(2)\delta x \delta z$
    \\$\Phi(1+2)= \delta x \delta z\left[ - C_{y}(1) + C_{y}(1)+ \pdv{C_y}{y}\delta y\right]= \pdv{C_y}{y}\delta x \delta y \delta z = \pdv{C_y}{y} \delta \tau $
    \\ Il flusso totale sull'elemento $\delta\Sigma_i$ è $\delta\Phi_i= \left( \pdv{C_x}{x}+\pdv{C_y}{y} +\pdv{C_z}{z}\right)\delta\tau_i= (\div \vb{C})\delta\tau_i$
    \\ $\sum_{i=1}\oint_{\delta\Sigma_i}\vb{C}\cdot\vu{n}\dd{\Sigma_i} = \sum_i \left( \div \vb{C} \right) \delta\tau_i = 
    \int_\tau \left( \div \vb{C} \right)\dd{\tau}$
\end{proof}

\section{Elettrostatica}
\subsection{Legge di Gauss }
Poniamo una sfera con centro nell'origine insieme a una carica $q$:
\begin{equation}
    \Phi(\vb{E})= \oint_\Sigma \vb{E}\cdot \vu{n} \; \dd{\Sigma} = \oint_\Sigma \frac{1}{4\pi\epsilon_0}\frac{q}{r_0^2}\vu{n}\cdot\vu{n}\dd{\Sigma}= 
    \frac{q}{4\pi\epsilon_0r_0^2}\oint_\Sigma \dd{\Sigma} = \frac{q}{\epsilon_0}
\end{equation}

Questo risultato si può generalizzare a una superficie qualsiasi con una carica all'interno.
\begin{equation}
    \delta\Phi(\vb{E})= \vb{E}\cdot\vu{n} \delta \Sigma = \frac{q}{4\pi\epsilon_0r_0^2}\vu{E}\cdot\vu{n} \delta\Sigma= 
     \frac{q}{4\pi\epsilon_0r_0^2}\cos \theta \delta\Sigma=  \frac{q}{4\pi\epsilon_0} \delta\Omega
\end{equation}
\begin{equation}
    \Phi(\vb{E})= \oint \dd{\Phi}(\vb{E})=\frac{q}{4\pi\epsilon_0} \oint\dd{\Omega} = \frac{q}{\epsilon_0}
\end{equation}
Se la carica è all'esterno della superficie il cono $\delta\Omega$ intercetta a coppie la superficie. 
\begin{equation}
    \delta\Phi(\vb{E}_2)= \vb{E}_2 \cdot\vu{n}_2 \delta\Sigma_2 =  \frac{q}{4\pi\epsilon_0} \delta\Omega
    \quad \quad \delta\Phi(\vb{E}_1)= \vb{E}_1 \cdot\vu{n}_1 \delta\Sigma_1 =  - \frac{q}{4\pi\epsilon_0} \delta\Omega
\end{equation}
\begin{equation}
    \delta\Phi(\Omega)= 0 \implies \Phi(\vb{E})= \oint \dd{\Phi}=0
\end{equation}

Nel caso generale con $n$ particelle interne e $m$ particelle esterne
\begin{equation}
    \vb{E}(p)= \sum_{k=1}^{n} \frac{q_k}{4\pi\epsilon_0} \frac{\vb{r}-\vb{r}_k}{\abs{\vb{r}-\vb{r}_k}^3} +
    \sum_{j=1}^{m} \frac{q_j}{4\pi\epsilon_0} \frac{\vb{r}-\vb{r}_j}{\abs{\vb{r}-\vb{r}_j}^3} 
\end{equation}
\begin{equation}
    \Phi(\vb{E})= \oint_\Sigma \vb{E}(p)\cdot\vu{n} \dd{\Sigma} = \oint_\Sigma \left[ \sum_k \dots + \sum_j \dots  \right]\cdot\vu{n} \dd{\Sigma} 
\end{equation}
\begin{equation}
    =\sum_{k=1}^{n} \oint_\Sigma\frac{q_k}{4\pi\epsilon_0} \frac{\vb{r}-\vb{r}_k}{\abs{\vb{r}-\vb{r}_k}^3} \cdot\vu{n}\dd{\Sigma} +
    \sum_{j=1}^{m} \oint_\Sigma\frac{q_j}{4\pi\epsilon_0} \frac{\vb{r}-\vb{r}_j}{\abs{\vb{r}-\vb{r}_j}^3}\cdot\vu{n} \dd{\Sigma} 
    = \sum_{k=1}^{n} \frac{q_k}{\epsilon_0} + 0 = \frac{Q_{interna}}{\epsilon_0}
\end{equation}

Consideriamo la densità di carica data da $\delta q = \rho(\vb{r})\delta\tau$
\begin{equation}
    \Phi(\vb{E})= \oint_\Sigma \vb{E}\cdot\vu{n} \dd{\Sigma}  \overset{T. di\, Gauss}{=} \int_\tau\left( \div \vb{E} \right) \dd{\tau}
    \overset{L. di\, Gauss}{=}\frac{1}{\epsilon_0}\int_\tau\rho \dd{\tau}
\end{equation}
\begin{equation}
    \implies \int_\tau \left( \div \vb{E}-\frac{\rho}{\epsilon_0} \right) \dd{\tau} = 0 \overset{\forall \tau}{\implies } \div \vb{E} = \frac{\rho}{\epsilon_0}
\end{equation}




\subsection{Forza Conservativa}
La forza di Coulomb è centrale.
\begin{definition}
    Un campo è di forze centrali $\vb{C}(\vb{r})$ con $O$ centro se $\exists O $ punto t.c. \\$\vb{C}(\vb{r})= C(r)\vu{r}\quad r = \abs{\overline{OP}}$
\end{definition}

\begin{definition}
    Un campo è conservativo se $\oint_\Gamma \vb{C}\cdot \dd{\vb{s}}= 0 \quad\forall\Gamma\leftrightarrow\int_{A}^{B}\vb{C}\cdot \dd{\vb{s}}= \Psi(B)-\Psi(A)$
\end{definition}
\begin{proof}
    $\Gamma= \Gamma_1\bigcup\Gamma_2$   \quad \quad         $0 = \oint_\Gamma \vb{C}\cdot \dd{\vb{s}}=\int_{A,\Gamma_1}^{B}\vb{C}\cdot \dd{\vb{s}}+\int_{B,\Gamma_2}^{A}\vb{C}\cdot \dd{\vb{s}}$
    \\$\implies \int_{A,\Gamma_1}^{B}\vb{C}\cdot \dd{\vb{s}}= \int_{A,\Gamma_2}^{B}\vb{C}\cdot \dd{\vb{s}}$
\end{proof}

\begin{proposition}
    Un campo centrale è conservativo 
\end{proposition}
\begin{proof}
    $\int_{A}^{B}\vb{C}(\vb{r})\cdot \dd{\vb{s}}=\int_{A}^{B}C(r)\vu{r}\cdot \dd{\vb{s}}\int_{r_A}^{r_B}C(r)\;dr= \Psi(r_B)-\Psi(r_A)$
\end{proof}
Per molteplici cariche il campo elettrico $\vb{E}(p)= \sum_{i=1}^{N} \vb{E}_i(p) $ non è centrale
\begin{equation}
    = \sum_{i=1}^{N}\frac{q_i}{4\pi\epsilon_0}\frac{(\vb{r}-\vb{r}_i)}{\abs{\vb{r}-\vb{r}_i}^3}\implies\int_{A,\Gamma}^{B}\vb{E}(p)\cdot \dd{\vb{s}}
    =\sum_{i=1}^{N}\left( \Psi_i(\vb{r}_B)-\Psi_i(\vb{r}_A) \right) 
\end{equation}
$\vb{E}$ è conservativo per quantità arbitrarie di cariche
\begin{equation}
    \oint_\Gamma \vb{E}\cdot \dd{\vb{s}} \overset{T.di\,Stokes}{=}\int_{\Sigma} \left( \curl \vb{E} \right)\cdot \vu{n}\dd{\Sigma}
    \overset{\forall\Sigma}{\implies}\curl\vb{E}=0
\end{equation}
Il campo elettrico in condizioni statiche è irrotazionale
\begin{definition}
    $\curl\vb{E}=0\implies \exists V | \vb{E}=- \grad V$ chiamato potenziale elettrostatico
\end{definition}

\subsection{Superfici equipotenziali}
\begin{definition}
    L'insieme dei punti dello spazio in cui il potenziale elettrostatico $V$ assume lo stesso valore
    è detto superficie equipotenziale.   
\end{definition}

\begin{proposition}
    Il campo $\vb{E}$ è ortogonale alle sue superfici equipotenziali
\end{proposition}
\begin{proof}
    $\Sigma$ equipotenziale $\implies\left[ \forall \vb{x} \in\Sigma \implies V(\vb{x}) \text{costante} \right]$
    \\$V(\vb{x}+\Delta\vb{x})-V(\vb{x})\simeq0=\pdv{V}{x}\Delta x+\pdv{V}{y}\Delta y+\pdv{V}{z}\Delta z= (\grad V)\cdot\Delta\vb{x}= - \vb{E}\cdot\Delta\vb{x}$
    \\$\implies-\vb{E}\cdot\Delta\vb{x}=0\implies\vb{E}\perp\Sigma$
\end{proof}

\newtheorem*{EP}{Equazione di Poisson}
\begin{EP}
    \begin{equation}
        \div \vb{E}= \frac{\rho}{\epsilon_0}\implies \laplacian V = -\frac{\rho}{\epsilon_0}
    \end{equation}
\end{EP}

\newtheorem*{EL}{Equazione di Laplace}
\begin{EL}
    \begin{equation}
        \laplacian V = 0
    \end{equation}
\end{EL}


%condizioni di contorno e soluzioni separabili
%da fare



Studiamo l'equazione di Laplace in una dimensione
\begin{equation}
    \dv[2]{\Phi}{x}=0 \implies\Phi(x) = mx + q \quad \text{Noto:} \quad\Phi(x) = \frac{\Phi(x+\Delta x)+\Phi(x-\Delta x)}{2}
\end{equation}
Posso estendere questo risultato a tre dimensioni
\begin{theorem}
    (Valore Medio)
    \begin{equation}
        \Phi(p)= \frac{1}{4\pi R^2}\oint_{\Sigma, Sfera} \Phi \;\dd{\Sigma}
    \end{equation}
\end{theorem}

\begin{proposition}
    Si stabilisce che in elettrostatica non si può verificare la condizione di equilibrio.
\end{proposition}
\begin{proof}
    Ipotizzo di avere un punto privo di carica in cui cercare $\vb{E}=0 \leftrightarrow$ equilibrio statico.
    Se è minimo, deve esserci una sfera $\Sigma$ intorno al punto con un potenziale t.c. $\forall\vb{x}\in\Sigma, V(\vb{x})>V(p)$;
    ma ciò è assurdo per il teorema del valore medio.
\end{proof}

\subsection{Esperimento di Rutherford}
%da fare

\subsection{Dipolo Elettrico}
Si tratta di due cariche opposte $+ q$ e $-q$ a una breve distanza $a$
\begin{equation}
    V(p)= \frac{q}{4\pi\epsilon_0}\left( \frac{1}{r_+}-\frac{1}{r_-} \right)= \frac{q}{4\pi\epsilon_0}\left( \frac{r_--r_+}{r_+r_-}\right)
\end{equation}
Se $r>>a$(far field) allora $r_+r_-\simeq r^2$, $r_--r_+\simeq a \cos \theta$
\begin{equation}
    \implies V(p)=  \frac{q}{4\pi\epsilon_0}\frac{a \cos \theta}{r^2}= \frac{1}{4\pi\epsilon_0}\frac{\vb{p}\cdot\vu{r}}{r^2}
\end{equation}

\begin{definition}
    $\vb{p}$ è detto momento del dipolo, è il vettore di modulo $aq$, diretto lungo l'asse del dipolo con verso dal negativo al positivo.

\end{definition}

\begin{equation}
    \vb{E}_r=E_r\vu{r}=-\pdv{V}{r}\vu{r}= \frac{2 p \cos \theta}{4\pi\epsilon_0r^3}\vu{r}
\end{equation}
\begin{equation}
    \vb{E}_\theta = -\frac{1}{r}\pdv{V}{\theta}\vu{\theta}= \frac{p \sin \theta}{4\pi\epsilon_0r^3}\vu{\theta}
\end{equation}
In generale $\vb{E}(p)=\vb{E}_r+\vb{E}_\theta$:
\begin{equation}
    \vb{E}(p)=\frac{3\vu{t}\left( \vb{p}\cdot\vu{t} \right)-\vb{p}}{4\pi\epsilon_0\abs{\vb{r}-\vb{r}'}^3} \quad \quad 
    \vu{t} = \frac{\vb{r}-\vb{r}'}{\abs{\vb{r}-\vb{r}'}}
\end{equation}

Se immergo un dipolo in un campo elettrico uniforme:
\begin{equation}
    \vb{F}_{tot} = \vb{F}_+ +\vb{F}_-= q\vb{E}-q\vb{E}=0  \quad \quad \implies \text{Baricentro fisso}
\end{equation}
\begin{equation}
    \vb{M}_{tot}= \vb{r}_+\times\vb{F}_+ + \vb{r}_-\times\vb{F}_-=\left( \vb{r}_++\vb{r}_- \right) \times\vb{F}=
     q\left( \vb{r}_++\vb{r}_- \right) \times\vb{E}= \vb{p}\times\vb{E}
\end{equation}
\begin{equation}
    U_{el}= -\vb{p}\cdot\vb{E}
\end{equation}
Se ho un campo dipendente da $x$:
\begin{equation}
    \vb{F}_{tot} = \vb{E}_+ +\vb{E}_-\simeq q\left( \vb{E}_- + \pdv{E}{x}a\vu{x} \right)-q\vb{E}_-= \vb{p}\pdv{E}{x}
\end{equation}















\end{document}