\documentclass[12pt,a4paper]{article}
\usepackage[utf8]{inputenc}
\usepackage[italian]{babel}
\usepackage{amsmath,amssymb,amsthm}
\usepackage{physics}  
\usepackage{graphicx}
\usepackage{hyperref}
\usepackage{geometry}
\usepackage{tikz}



\geometry{margin=2.5cm}

\setlength{\parindent}{0pt}
%\setcounter{tocdepth}{2}

\usetikzlibrary{arrows.meta} 
\usetikzlibrary{patterns}


\theoremstyle{definition}
\newtheorem{theorem}{Teorema}
\newtheorem{definition}{Definizione}
\newtheorem{example}{Esempio}
\newtheorem{proposition}{Proposizione}
\newtheorem{lemma}{Lemma}
\newtheorem{remark}{Nota}

\title{Fisica 2 - Appunti}
\author{Davide Bagnoli}
\date{1 semestre 2024/2025}




\begin{document}
\maketitle
\tableofcontents
\newpage

\section{Analisi vettoriale}
\subsection{Note}
\begin{proposition}
    \begin{equation*}
    \div(\curl \vb{A})= 0
    \end{equation*}
\end{proposition}
\begin{proof}
$
\grad_{x}\vdot(\curl{\vb{A}})_{x}=\pdv{x}(\pdv{A_z}{y}-\pdv{A_y}{z})$
se $\vb{A}$ è $\mathcal{C}^{2}$ le derivate si scambiano e si elidono con gli altri termini.
\end{proof}

\begin{proposition}
    \begin{equation*}
    \curl(\grad \Psi)=0
    \end{equation*}
\end{proposition}

\begin{proof}
 $(\curl(\grad\Psi))_{x} = \vu{x}\left( \pdv{\Psi}{y}{z} -\pdv{\Psi}{z}{y}\right)=0$ se $\vb{A}$ è $\mathcal{C}^{2}$   
\end{proof}

\begin{theorem}
    $\forall$p $\div \vb{A}(p)= 0 \implies \exists\vb{C} $ campo vettoriale t.c. $\vb{A} = \curl \vb{C}$
\end{theorem}

\begin{theorem}
    $\forall$p $\curl \vb{A}(p)= 0 \implies \exists\Psi $ campo scalare t.c. $\vb{A} = \grad \Psi$
\end{theorem}

\newtheorem*{TS}{Teorema di Strokes}
\begin{TS}
    \begin{equation*}
    \oint_{\Gamma}\vb{C}\cdot \dd{\vb{s}}= \int_{\Sigma}(\curl \vb{C})\cdot \vu{n} \dd{\Sigma}
    \end{equation*}
\end{TS}

\begin{proof}
    Divido la superficie in quadrati abbastanza piccoli da considerarsi piani
    $\oint_{\Gamma}\vb{C}\cdot \dd{\vb{s}}= \sum_{i=1}\oint_{\Gamma_{i}}\vb{C}\cdot \dd{\vb{s}}$\\
    $\oint_{\Gamma_{i}}\vb{C}\cdot \dd{\vb{s}}= C_{x}(1)\delta x +C_{y}(2)\delta y -C_{x}(3)\delta x -C_{x}(4)\delta y $ dove $C_{x}$ e $C_{y}$ sono costanti.
    \\$[C_{x}(1)-C_{x}(3)]\delta x = [C_{x}(1)-C_{x}(1)-\pdv{C_x}{y}\delta y]\delta x \simeq -\pdv{C_x}{y}\delta y\delta x$
    \\$[C_{y}(2)-C_{y}(4)]\delta x = [C_{y}(2)-C_{y}(2)-\left( -\pdv{C_y}{x} \right)\delta x]\delta y \simeq +\pdv{C_y}{x}\delta x\delta y$
    \\$ \sum_{i=1}\oint_{\Gamma_{i}}\vb{C}\cdot \dd{\vb{s}}= \sum_{i=1}\left( +\pdv{C_y}{x} -\pdv{C_x}{y}\right) \delta y_{i}\delta x_{i}
    = \int_{\Sigma}\left( +\pdv{C_y}{x} -\pdv{C_x}{y}\right) \dd{\Sigma} =  \int_{\Sigma}\left( \curl \vb{C}\right) \dd{\Sigma}  $
\end{proof}

\newtheorem*{TG}{Teorema di Gauss}
\begin{TG}
    \begin{equation*}
        \oint_{\Sigma} \vb{C}\cdot\vu{n}\dd{\Sigma}= \int_{V}\left( \div\vb{C} \right) \dd{\tau}
    \end{equation*}
\end{TG}
\begin{proof}
    Posso dividere il volume della superficie in tanti piccoli cubetti
    \\ $\oint_{\Sigma} \vb{C}\cdot\vu{n}\dd{\Sigma}= \sum_{i=1} \oint_{\Sigma_{i}} \vb{C}\cdot\vu{n} \dd{\Sigma_{i}}$ 
    \\$\Phi(1) = -\int C_{y}\dd{x}\dd{z} =- C_{y}(1)\delta x \delta z$ \quad
    $\Phi(2) =  C_{y}(2)\delta x \delta z$
    \\$\Phi(1+2)= \delta x \delta z\left[ - C_{y}(1) + C_{y}(1)+ \pdv{C_y}{y}\delta y\right]= \pdv{C_y}{y}\delta x \delta y \delta z = \pdv{C_y}{y} \delta \tau $
    \\ Il flusso totale sull'elemento $\delta\Sigma_i$ è $\delta\Phi_i= \left( \pdv{C_x}{x}+\pdv{C_y}{y} +\pdv{C_z}{z}\right)\delta\tau_i= (\div \vb{C})\delta\tau_i$
    \\ $\sum_{i=1}\oint_{\delta\Sigma_i}\vb{C}\cdot\vu{n}\dd{\Sigma_i} = \sum_i \left( \div \vb{C} \right) \delta\tau_i = 
    \int_\tau \left( \div \vb{C} \right)\dd{\tau}$
\end{proof}

\section{Elettrostatica}
\subsection{Legge di Gauss }
Poniamo una sfera con centro nell'origine insieme a una carica $q$:
\begin{equation*}
    \Phi(\vb{E})= \oint_\Sigma \vb{E}\cdot \vu{n} \; \dd{\Sigma} = \oint_\Sigma \frac{1}{4\pi\epsilon_0}\frac{q}{r_0^2}\vu{n}\cdot\vu{n}\dd{\Sigma}= 
    \frac{q}{4\pi\epsilon_0r_0^2}\oint_\Sigma \dd{\Sigma} = \frac{q}{\epsilon_0}
\end{equation*}

Questo risultato si può generalizzare a una superficie qualsiasi con una carica all'interno.
\begin{equation*}
    \delta\Phi(\vb{E})= \vb{E}\cdot\vu{n} \delta \Sigma = \frac{q}{4\pi\epsilon_0r_0^2}\vu{E}\cdot\vu{n} \delta\Sigma= 
     \frac{q}{4\pi\epsilon_0r_0^2}\cos \theta \delta\Sigma=  \frac{q}{4\pi\epsilon_0} \delta\Omega
\end{equation*}
\begin{equation*}
    \Phi(\vb{E})= \oint \dd{\Phi}(\vb{E})=\frac{q}{4\pi\epsilon_0} \oint\dd{\Omega} = \frac{q}{\epsilon_0}
\end{equation*}
Se la carica è all'esterno della superficie il cono $\delta\Omega$ intercetta a coppie la superficie. 
\begin{equation*}
    \delta\Phi(\vb{E}_2)= \vb{E}_2 \cdot\vu{n}_2 \delta\Sigma_2 =  \frac{q}{4\pi\epsilon_0} \delta\Omega
    \quad \quad \delta\Phi(\vb{E}_1)= \vb{E}_1 \cdot\vu{n}_1 \delta\Sigma_1 =  - \frac{q}{4\pi\epsilon_0} \delta\Omega
\end{equation*}
\begin{equation*}
    \delta\Phi(\Omega)= 0 \implies \Phi(\vb{E})= \oint \dd{\Phi}=0
\end{equation*}

Nel caso generale con $n$ particelle interne e $m$ particelle esterne
\begin{equation*}
    \vb{E}(p)= \sum_{k=1}^{n} \frac{q_k}{4\pi\epsilon_0} \frac{\vb{r}-\vb{r}_k}{\abs{\vb{r}-\vb{r}_k}^3} +
    \sum_{j=1}^{m} \frac{q_j}{4\pi\epsilon_0} \frac{\vb{r}-\vb{r}_j}{\abs{\vb{r}-\vb{r}_j}^3} 
\end{equation*}
\begin{equation*}
    \Phi(\vb{E})= \oint_\Sigma \vb{E}(p)\cdot\vu{n} \dd{\Sigma} = \oint_\Sigma \left[ \sum_k \dots + \sum_j \dots  \right]\cdot\vu{n} \dd{\Sigma} 
\end{equation*}
\begin{equation*}
    =\sum_{k=1}^{n} \oint_\Sigma\frac{q_k}{4\pi\epsilon_0} \frac{\vb{r}-\vb{r}_k}{\abs{\vb{r}-\vb{r}_k}^3} \cdot\vu{n}\dd{\Sigma} +
    \sum_{j=1}^{m} \oint_\Sigma\frac{q_j}{4\pi\epsilon_0} \frac{\vb{r}-\vb{r}_j}{\abs{\vb{r}-\vb{r}_j}^3}\cdot\vu{n} \dd{\Sigma} 
    = \sum_{k=1}^{n} \frac{q_k}{\epsilon_0} + 0 = \frac{Q_{interna}}{\epsilon_0}
\end{equation*}

Consideriamo la densità di carica data da $\delta q = \rho(\vb{r})\delta\tau$
\begin{equation*}
    \Phi(\vb{E})= \oint_\Sigma \vb{E}\cdot\vu{n} \dd{\Sigma}  \overset{T. di\, Gauss}{=} \int_\tau\left( \div \vb{E} \right) \dd{\tau}
    \overset{L. di\, Gauss}{=}\frac{1}{\epsilon_0}\int_\tau\rho \dd{\tau}
\end{equation*}
\begin{equation*}
    \implies \int_\tau \left( \div \vb{E}-\frac{\rho}{\epsilon_0} \right) \dd{\tau} = 0 \overset{\forall \tau}{\implies } \div \vb{E} = \frac{\rho}{\epsilon_0}
\end{equation*}




\subsection{Forza Conservativa}
La forza di Coulomb è centrale.
\begin{definition}
    Un campo è di forze centrali $\vb{C}(\vb{r})$ con $O$ centro se $\exists O $ punto t.c. \\$\vb{C}(\vb{r})= C(r)\vu{r}\quad r = \abs{\overline{OP}}$
\end{definition}

\begin{definition}
    Un campo è conservativo se $\oint_\Gamma \vb{C}\cdot \dd{\vb{s}}= 0 \quad\forall\Gamma\leftrightarrow\int_{A}^{B}\vb{C}\cdot \dd{\vb{s}}= \Psi(B)-\Psi(A)$
\end{definition}
\begin{proof}
    $\Gamma= \Gamma_1\bigcup\Gamma_2$   \quad \quad         $0 = \oint_\Gamma \vb{C}\cdot \dd{\vb{s}}=\int_{A,\Gamma_1}^{B}\vb{C}\cdot \dd{\vb{s}}+\int_{B,\Gamma_2}^{A}\vb{C}\cdot \dd{\vb{s}}$
    \\$\implies \int_{A,\Gamma_1}^{B}\vb{C}\cdot \dd{\vb{s}}= \int_{A,\Gamma_2}^{B}\vb{C}\cdot \dd{\vb{s}}$
\end{proof}

\begin{proposition}
    Un campo centrale è conservativo 
\end{proposition}
\begin{proof}
    $\int_{A}^{B}\vb{C}(\vb{r})\cdot \dd{\vb{s}}=\int_{A}^{B}C(r)\vu{r}\cdot \dd{\vb{s}}\int_{r_A}^{r_B}C(r)\;dr= \Psi(r_B)-\Psi(r_A)$
\end{proof}
Per molteplici cariche il campo elettrico $\vb{E}(p)= \sum_{i=1}^{N} \vb{E}_i(p) $ non è centrale
\begin{equation*}
    = \sum_{i=1}^{N}\frac{q_i}{4\pi\epsilon_0}\frac{(\vb{r}-\vb{r}_i)}{\abs{\vb{r}-\vb{r}_i}^3}\implies\int_{A,\Gamma}^{B}\vb{E}(p)\cdot \dd{\vb{s}}
    =\sum_{i=1}^{N}\left( \Psi_i(\vb{r}_B)-\Psi_i(\vb{r}_A) \right) 
\end{equation*}
$\vb{E}$ è conservativo per quantità arbitrarie di cariche
\begin{equation*}
    \oint_\Gamma \vb{E}\cdot \dd{\vb{s}} \overset{T.di\,Stokes}{=}\int_{\Sigma} \left( \curl \vb{E} \right)\cdot \vu{n}\dd{\Sigma}
    \overset{\forall\Sigma}{\implies}\curl\vb{E}=0
\end{equation*}
Il campo elettrico in condizioni statiche è irrotazionale
\begin{definition}
    $\curl\vb{E}=0\implies \exists V | \vb{E}=- \grad V$ chiamato potenziale elettrostatico
\end{definition}

\subsection{Superfici equipotenziali}
\begin{definition}
    L'insieme dei punti dello spazio in cui il potenziale elettrostatico $V$ assume lo stesso valore
    è detto superficie equipotenziale.   
\end{definition}

\begin{proposition}
    Il campo $\vb{E}$ è ortogonale alle sue superfici equipotenziali
\end{proposition}
\begin{proof}
    $\Sigma$ equipotenziale $\implies\left[ \forall \vb{x} \in\Sigma \implies V(\vb{x}) \text{costante} \right]$
    \\$V(\vb{x}+\Delta\vb{x})-V(\vb{x})\simeq0=\pdv{V}{x}\Delta x+\pdv{V}{y}\Delta y+\pdv{V}{z}\Delta z= (\grad V)\cdot\Delta\vb{x}= - \vb{E}\cdot\Delta\vb{x}$
    \\$\implies-\vb{E}\cdot\Delta\vb{x}=0\implies\vb{E}\perp\Sigma$
\end{proof}

\newtheorem*{EP}{Equazione di Poisson}
\begin{EP}
    \begin{equation*}
        \div \vb{E}= \frac{\rho}{\epsilon_0}\implies \laplacian V = -\frac{\rho}{\epsilon_0}
    \end{equation*}
\end{EP}

\newtheorem*{EL}{Equazione di Laplace}
\begin{EL}
    \begin{equation*}
        \laplacian V = 0
    \end{equation*}
\end{EL}


%condizioni di contorno e soluzioni separabili
%da fare



Studiamo l'equazione di Laplace in una dimensione
\begin{equation*}
    \dv[2]{\Phi}{x}=0 \implies\Phi(x) = mx + q \quad \text{Noto:} \quad\Phi(x) = \frac{\Phi(x+\Delta x)+\Phi(x-\Delta x)}{2}
\end{equation*}
Posso estendere questo risultato a tre dimensioni
\begin{theorem}
    (Valore Medio)
    \begin{equation*}
        \Phi(p)= \frac{1}{4\pi R^2}\oint_{\Sigma, Sfera} \Phi \;\dd{\Sigma}
    \end{equation*}
\end{theorem}

\begin{proposition}
    Si stabilisce che in elettrostatica non si può verificare la condizione di equilibrio.
\end{proposition}
\begin{proof}
    Ipotizzo di avere un punto privo di carica in cui cercare $\vb{E}=0 \leftrightarrow$ equilibrio statico.
    Se è minimo, deve esserci una sfera $\Sigma$ intorno al punto con un potenziale t.c. $\forall\vb{x}\in\Sigma, V(\vb{x})>V(p)$;
    ma ciò è assurdo per il teorema del valore medio.
\end{proof}

\subsection{Esperimento di Rutherford}
%da fare

\subsection{Dipolo Elettrico}
Si tratta di due cariche opposte $+ q$ e $-q$ a una breve distanza $a$
\begin{equation*}
    V(p)= \frac{q}{4\pi\epsilon_0}\left( \frac{1}{r_+}-\frac{1}{r_-} \right)= \frac{q}{4\pi\epsilon_0}\left( \frac{r_--r_+}{r_+r_-}\right)
\end{equation*}
Se $r>>a$(far field) allora $r_+r_-\simeq r^2$, $r_--r_+\simeq a \cos \theta$
\begin{equation*}
    \implies V(p)=  \frac{q}{4\pi\epsilon_0}\frac{a \cos \theta}{r^2}= \frac{1}{4\pi\epsilon_0}\frac{\vb{p}\cdot\vu{r}}{r^2}
\end{equation*}

\begin{definition}
    $\vb{p}$ è detto momento del dipolo, è il vettore di modulo $aq$, diretto lungo l'asse del dipolo con verso dal negativo al positivo.

\end{definition}

\begin{equation*}
    \vb{E}_r=E_r\vu{r}=-\pdv{V}{r}\vu{r}= \frac{2 p \cos \theta}{4\pi\epsilon_0r^3}\vu{r}
\end{equation*}
\begin{equation*}
    \vb{E}_\theta = -\frac{1}{r}\pdv{V}{\theta}\vu{\theta}= \frac{p \sin \theta}{4\pi\epsilon_0r^3}\vu{\theta}
\end{equation*}
In generale $\vb{E}(p)=\vb{E}_r+\vb{E}_\theta$:
\begin{equation*}
    \vb{E}(p)=\frac{3\vu{t}\left( \vb{p}\cdot\vu{t} \right)-\vb{p}}{4\pi\epsilon_0\abs{\vb{r}-\vb{r}'}^3} \quad \quad 
    \vu{t} = \frac{\vb{r}-\vb{r}'}{\abs{\vb{r}-\vb{r}'}}
\end{equation*}

Se immergo un dipolo in un campo elettrico uniforme:
\begin{equation*}
    \vb{F}_{tot} = \vb{F}_+ +\vb{F}_-= q\vb{E}-q\vb{E}=0  \quad \quad \implies \text{Baricentro fisso}
\end{equation*}
\begin{equation*}
    \vb{M}_{tot}= \vb{r}_+\times\vb{F}_+ + \vb{r}_-\times\vb{F}_-=\left( \vb{r}_++\vb{r}_- \right) \times\vb{F}=
     q\left( \vb{r}_++\vb{r}_- \right) \times\vb{E}= \vb{p}\times\vb{E}
\end{equation*}
\begin{equation*}
    U_{el}= -\vb{p}\cdot\vb{E}
\end{equation*}
Se ho un campo dipendente da $x$:
\begin{equation*}
    \vb{F}_{tot} = \vb{E}_+ +\vb{E}_-\simeq q\left( \vb{E}_- + \pdv{E}{x}a\vu{x} \right)-q\vb{E}_-= \vb{p}\pdv{E}{x}
\end{equation*}
\subsubsection{Espansione Multipolo}
Possiamo immaginare un qualsiasi corpo come un insieme di cariche puntuali distribuite in una regione limitata di spazio.
Il potenziale attorno a questo corpo è approssimabile da:
\begin{equation*}
    V(p)= \frac{Q_{tot}}{4\pi\epsilon_0 r}+ \frac{\vb{p}\cdot\vu{r}}{4\pi\epsilon_0r^2}+ \frac{1}{4\pi\epsilon_0r^3}
    \sum_{i=1}^{N} q_i r_i^2\frac{(3 \cos\theta_i-1)}{2} + o(r^{-4})
\end{equation*}
\begin{proof}
    \begin{equation*}
        V(p)= \sum_{i=1}^{N} \frac{q_i}{4\pi\epsilon_0\abs{\vb{r}-\vb{r}_i}} 
        \quad \quad t_i^2= (\vb{r}-\vb{r}_i)^2= r^2+r_i^2-2rr_i\cos \theta_i
    \end{equation*}
    Applicando approssimazione al far field $(r>>a)$
    \begin{equation*}
        \frac{1}{t_i}= \frac{1}{\sqrt{r^2+r_i^2-2rr_i\cos \theta_i}}= \frac{1}{r}\frac{1}{\sqrt{1-2\frac{r_i}{r}\cos \theta_i + \frac{r_i^2}{r^2}}}
        \simeq\frac{1}{r}\left( 1+ \frac{r_i}{r}\cos \theta_i +o(\frac{r_i^2}{r^2})\right)
    \end{equation*}
    \begin{equation*}
        \implies V(p)= \frac{1}{4\pi\epsilon_0}\sum_{i=1}^{N} \frac{q_i}{t_i}= \frac{1}{4\pi\epsilon_0} \left[ \frac{\sum q_i}{r}+\frac{\sum q_ir_i \cos \theta_i}{r^2} + o(\frac{1}{r^3})\right]
    \end{equation*}
\end{proof}

\subsection{Conduttori}
\begin{definition}
    Se un materiale ha cariche che riescono a muoversi se sottoposte a un campo elettrico, si dice conduttore. Altrimenti, è detto dielettrico.
\end{definition}

\subsubsection{Conduttori in equilibrio statico}
Se il conduttore è in equilibrio statico, allora: $\vb{E}=0$; altrimenti le cariche interne che sono libere, lo seguirebbero.
Posso spostare cariche esterne, ma subito dopo la perturbazione viene ripristinato l'equilibrio interno nell'ordine di picosecondi.
\\
$\vb{E}=0 \implies Q_{int} = 0\implies \forall\Sigma$ interna non può contenere carica, la carica si trova solo sui primi stati atomici
$\sigma= \frac{\dd{q}}{\dd{\Sigma}}$.
\\
$\vb{E}=0 \implies  V = $ costante, il campo è ortogonale alla superficie del conduttore. 
\\Se considero un elemento infinitesimo $\dd{\Sigma}$ della superficie del conduttore, sulla parte interna non c'è campo elettrico e su quella esterna:
\begin{equation*}
    \Phi(\vb{E})= \frac{Q_{int}}{\epsilon_0}= \frac{\dd{Q}}{\epsilon_0}= \frac{\sigma}{\epsilon_0}\dd{\Sigma}= E\dd{\Sigma}\implies E = \frac{\sigma}{\epsilon_0}
\end{equation*}
Se ho un foglio carico ho una situazione simile:
\begin{equation*}
    \Phi(\vb{E}) = E \cdot 2\dd{\Sigma}= \frac{\sigma}{\epsilon_0}\dd{\Sigma}\implies E = \frac{\sigma}{2\epsilon_0}
\end{equation*}

Consideriamo ora una cavità vuota all'interno del conduttore, una superficie $\Sigma $ che contiene la cavità  e un circuito $\Gamma$ che attravera la cavità e il conduttore:
\\$\Phi_\Sigma(\vb{E})= 0 \implies Q_{int} = 0 $ la carica totale nella cavità è zero.
\\$\Gamma(E)= 0 = \int_{\Gamma,int}0 \cdot\dd{\vb{s}}+\int_{\Gamma,cav}\vb{E}_{cav}\cdot\dd{\vb{s}}\implies E_{cav}= 0, \quad
\sigma_{cav}=0$

Se la cavità all'interno della cavità ho una carica $q$, sulla superficie interna del conduttore avrò una carica $-q$.

Variazioni all'interno della cavità non avranno conseguenze esterne e viceversa, una gabbia conduttrice, detta \textit{gabbia di Faraday},
permette di isolare sistemi elettrostatici.

\subsubsection{Carica immagine}
Date due cariche opposte a una certa distanza, consideriamo una superficie equipotenziale $\Sigma$ con $V(\vb{r})= V_A$ e prendiamo un foglio 
uguale a $\Sigma$. Se manteniamo il potenziale del conduttore $V = V_A$ allora potrò metterlo su $\Sigma$ senza cambiare la situazione, da quel
punto sarà irrilevante la presenza della carica interna: posso lavorare solo con il conduttore cavo o pieno.
\begin{proposition}
    Posso sostituire a una carica un conduttore congruente a una sua superficie equipotenziale e viceversa.
\end{proposition}

In generale se ho un sistema con $N$ conduttori questi avranno una certa carica $Q_k$ e un certo potenziale $V_k$ che dipendono uno dall'altro
\begin{equation*}
    V_k = \sum_{i=1}^{N} a_{ki}Q_i \quad \quad Q_k = \sum_{i=1}^{N} C_{ki}V_i
\end{equation*}

I $C_{ki}$ sono detti \textit{coefficienti d'induzione}, $C_{kk}$ è detta \textit{capacità}.

\subsection{Condensatori}
\begin{definition}
    Un \textit{condensatore} è un sistema formato da coppie di armature conduttrici
\end{definition}

\subsubsection{Gusci sferici}
Consideriamo due gusci sferici concentrici il primo con carica $+q$ e raggio $R_1$ e il secondo con carica $-q$ e raggio $R_2 > R_1$:
\begin{equation*}
    E(r<R_1)=0\qquad E(r>R_2)=0\qquad E(r \in [R_1,R_2])= \frac{q}{4\pi\epsilon_0r^2}
\end{equation*}
\begin{equation*}
    \Delta V = V_1-V_2= \int_{2}^{1}-\vb{E}\cdot\dd{\vb{s}}= \int_{1}^{2}\vb{E}\cdot\dd{\vb{s}}= 
    \int_{1}^{2}\frac{q}{4\pi\epsilon_0}\frac{\vu{r}}{r^2}\cdot\dd{\vb{s}} = \int_{R_1}^{R_2}\frac{q}{4\pi\epsilon_0}\frac{\dd{r}}{r^2}
\end{equation*}
\begin{equation*}
    \implies V = \frac{q}{4\pi\epsilon_0}\eval{\frac{1}{r}}_{R_1}^{R_2}= \frac{q}{4\pi\epsilon_0}\left( \frac{1}{R_1}-\frac{1}{R_2} \right)
    = \frac{q}{4\pi\epsilon_0}\frac{R_2-R_1}{R_1R_2}
\end{equation*}
\begin{equation*}
    C = 4\pi\epsilon_0\frac{R_1R_2}{R_2-R_1}
\end{equation*}

\subsubsection{Condensatori piani}
Consideriamo due armature piane di superficie $\Sigma$ e carica opposta $+q$ e $-q$ poste a una distanza $d<<\Sigma$, trascurando gli effetti di bordo si ottiene:
\begin{equation*}
    E(\text{tra le armature})= \frac{\sigma}{\epsilon_0}\implies V = \frac{\sigma}{\epsilon_0}d = \frac{q}{\epsilon_0\Sigma}d 
    \implies C = \epsilon_0\frac{\Sigma}{d}
\end{equation*}
\subsubsection{Serie e parallelo}
Consideriamo due condensatori $C_1$ e $C_2$ collegati in parallelo, la caduta di potenziale $\Delta V$ è la stessa ai due condensatori.
La capacità $C_{eq}$ di un singolo condensatore equivalente al nostro caso, si ottiene:
\begin{equation*}
    Q_1= C_1\Delta V\qquad Q_2 = C_2 \Delta V \qquad C_{eq}\Delta V = Q_1 +Q_2 \implies C_{eq}= C_1 + C_2
\end{equation*}

Consideriamo ora due condensatori $C_1$ e $C_2$ collegati in serie, la carica $Q$ è la stessa per i due condensatori:
\begin{equation*}
    Q = C_1 \Delta V_1\qquad Q= C_2 \Delta V_2 \implies \Delta V = \Delta V_1 + \Delta V_2 = \frac{Q}{C_1} + \frac{Q}{C_2}= 
    Q\left( \frac{1}{C_1}+\frac{1}{C_2} \right)
\end{equation*}
\begin{equation*}
    \Delta V = Q\left( \frac{1}{C_1}+\frac{1}{C_2} \right) = \frac{Q}{C_{eq}} \implies C_{eq} = \left( \frac{1}{C_1}+ \frac{1}{C_2} \right)^{-1}
\end{equation*}

\subsubsection{Energia elettrostatica}
Consideriamo il condensatore inizialmente scarico, lo si carica portando le cariche da un lato all'altro utilizzando un circuito. 
Dopo aver portato una carica sarà più difficile spostare un altra carica nello stesso modo essendo aumentato il potenziale.
\begin{equation*}
    \dd{W} = \dd{q} \Delta V(q) = \dd{q} \frac{q}{C} \implies W = \int_{0}^{Q} \dd{q} \frac{q}{C}= \frac{1}{2}\frac{Q^2}{C}= \frac{1}{2}CV^2
\end{equation*}
\begin{equation*}
    U_{E} = \frac{1}{2}\frac{Q^2}{C}= \frac{1}{2}CV^2 = \frac{1}{2}QV \qquad\text{Energia potenziale elettrostatica}
\end{equation*}
\begin{equation*}
    \overset{piano}{=} \frac{1}{2}\epsilon_0\frac{\Sigma}{d}V^2= \frac{1}{2}\epsilon_0\Sigma d \frac{V^2}{d^2}= \frac{1}{2} \epsilon_0\tau E^2
    \implies \mu_E = \frac{1}{2}\epsilon_0E^2
\end{equation*}
Abbiamo ricavato un'espressione della \textit{densità energetica del campo elettrico}, cerchiamo di dimostrarla formalmente:
\begin{proof}
    Partiamo dalla nota $U_{E}= \frac{1}{4\pi\epsilon_0}\frac{1}{2}\sum_{i\neq j}\frac{q_iq_j}{\abs{\vb{r}_i-\vb{r}_j}} $ passando alla sua forma continua:
    \begin{equation*}
        U_E= \frac{1}{2}\frac{1}{4\pi\epsilon_0}\int_\tau\int_\tau\frac{\rho(\vb{r})\rho(\vb{r}')}{\abs{\vb{r}-\vb{r}'}}\dd{\tau}\dd{\tau'}\implies
        U_E= \frac{1}{2}\int_\tau V(\vb{r})\rho(\vb{r}) \dd{\tau}
    \end{equation*}
    Utilizzando l'equazione di Poisson:
    \begin{equation*}
        \laplacian V = -\frac{\rho}{\epsilon_0}\implies U_E = -\frac{1}{2}\epsilon_0\int_\tau V \left( \laplacian{V} \right)\dd{\tau}
    \end{equation*}
    \begin{equation*}
        V \laplacian{V}= V \left(\sum_{i=1}^{3} \pdv[2]{V}{x_i}\right)= \sum_{i=1}^{3}\pdv{}{x_i}\left( V \pdv{V}{x_i} \right)- \left( \pdv{V}{x_i} \right)\left( \pdv{V}{x_i} \right)
        = \div \left( V \grad{V} \right)- \left( \grad{V} \right)\left( \grad{V} \right) 
    \end{equation*}
    \begin{equation*}
       \implies U_E = \frac{1}{2} \epsilon_0\int_\tau(\grad{v})(\grad{V}) \dd{\tau}-\frac{1}{2}\epsilon_0\int_\tau\div(V\grad{V})\dd{\tau}
    \end{equation*}
    Concentriamoci sul secondo pezzo dell'equazione:
    \begin{equation*}
        \int_\tau\div(V\grad{V})\dd{\tau} \overset{Gauss}{=} \int_\Sigma\left( V\grad{V} \right)\cdot\vu{n}\dd{\Sigma}
    \end{equation*}
    Prendo come volume $\tau$ una sfera molto grande $(R\rightarrow+\infty)$:
    \begin{equation*}
        V \sim \frac{1}{R}  \quad\grad{V}= \frac{1}{R^2}\implies \int_\Sigma \left( V\grad{V} \right)\cdot\vu{n}\dd{\Sigma}
        \sim \frac{1}{R}\frac{1}{R^2}4\pi R^2 \sim \frac{1}{R}\rightarrow 0 
    \end{equation*}
    \begin{equation*}
        U_E = \frac{1}{2}\epsilon_0\int_{\tau}\left( \grad{V} \right)\left( \grad{V} \right)\dd{\tau}= \frac{1}{2}\epsilon_0\int_{\tau}
        E^2 \dd{\tau}= \int_\tau  \mu_E \dd{\tau}
    \end{equation*}
\end{proof}

\subsection{Dielettrici e polarizzazione}
Nei dielettrici le cariche non sono libere, se inserisco un materiale dielettrico all'interno di un condesantore 
il potenziale diminuisce di una certa costante $\kappa_e$ detta \textit{costante dielettrica relativa}.

Per capire perché succede consideriamo innanzitutto una molecola o atomo non polare in mezzo alle piastre del condensatore,
sottoponendolo a un campo elettrico induco nella molecola un momento di dipolo, questo fenomeno si dice \textit{polarizzazione elettronica}.
\\Se prendo invece una molecola polare inizialmente il dipolo naturale sarà orientato casualmente, introducendolo in un campo elettrico
il dipolo si orienterà di conseguenza, (\textit{polarizzazione per orientamento})


In un dielettrico abbiamo molti di questi dipoli $\vb{p}_E$, per ogni elemento infinitesimo di volume $\dd{\tau}$ abbiamo $\dd{\vb{p}}= n \vb{p}_E\dd{\tau}$.
Possiamo così definire il \textit{vettore polarizzazione} $\vb{P}=\dv{\vb{p}}{\tau}$
\\Per i materiali non ferromagnetici la polarizzazione è funzione del campo elettrico $\vb{P}= \vb{P}(\vb{E})$. Per $E$ abbastanza piccoli
si può suppore una relazione lineare tra le due:
\begin{equation*}
    \vb{P}= \epsilon_0\chi_E\vb{E}
\end{equation*}
$\chi_E$ si chiama \textit{suscettività dielettrica}, in generale è un tensore ma se ci limitiamo a materiali isotropi è uno scalare.
Inoltre, considereremo solo materiali omogenei in cui $\chi_E$ non dipende dalla posizione.

Il modulo di $\vb{P}$ è uguale alla densità di carica indotta sulle superfici del dielettrico $\sigma_{pol}$
\begin{equation*}
    p_{tot}= Q_{pol}d= \sigma_{pol}\Sigma d= \sigma_{pol}\tau \implies \sigma_{pol}= \frac{p_{tot}}{\tau}= P
\end{equation*}

Quanto vale $\vb{E}$ all'interno del dielettrico?
\begin{equation*}
    E_0= \frac{\sigma_{free}}{\epsilon_0}\qquad E = \frac{\sigma_{free}-\sigma_{pol}}{\epsilon_0}=\frac{\sigma_{free}-\vb{P}}{\epsilon_0}
    = \frac{\sigma_{free}}{\epsilon_0}-\chi_EE 
\end{equation*}
\begin{equation*}
    \implies E(1+ \chi_E)= \frac{\sigma_{free}}{\epsilon_0}= E_0 \implies E = \frac{E_0}{1+ \chi_E}\qquad \kappa_E= 1+ \chi_E
\end{equation*}

Consideriamo il campo di polarizzazione $\vb{P}(\vb{r})$. La carica totale legata all'interno di un volume $\tau$ è data da:

\begin{equation*}
    Q_{pol} = -\oint_{\Sigma} \vb{P} \cdot \dd{\vb{\Sigma}} \overset{Gauss}{=} -\int_{\tau} \div \vb{P} \dd{\tau}
\end{equation*}
Per definizione \( Q_{pol} = \int_{V} \rho_{\text{pol}} \dd{\tau} \), quindi:
\begin{equation*}
    \int_{V} \rho_{pol} \dd{\tau} = -\int_{V} \div \vb{P} \dd{\tau}
    \implies \rho_{pol} = -\div \vb{P}
\end{equation*}
Notiamo che:
\begin{equation*}
    \div\vb{E}=\frac{\rho}{\epsilon_0}= \left( \rho_{free} +\rho_{pol} \right) \frac{1}{\epsilon_0}\implies \div(\vb{E}+\frac{\vb{P}}{\epsilon_0})= \frac{\rho_{free}}{\epsilon_0}
\end{equation*}
Possiamo definire il \textit{vettore spostamento} $\vb{D}$ tale che:
\begin{equation*}
    \vb{D}= \epsilon_0\vb{E} + \vb{P} \implies \div \vb{D} = \rho_{free}
\end{equation*}

\subsubsection{Campi elettrici nelle zone di interfaccia}
Consideriamo una superficie in cui due dielettrici differenti $(\kappa_1, \kappa_2)$ si incontrano, detta interfaccia. 
Si consideri un circuito rettangolare $\Gamma$ lungo l'interfaccia largo $l$:
\begin{equation*}
    \int_\Gamma \vb{E}\cdot\dd{\vb{s}}= \int_\Gamma \vb{E}_\parallel\cdot\dd{\vb{s}}= 0 = E_{1,\parallel}l-E_{2,\parallel}l \implies \vb{E}_{1,\parallel}= \vb{E}_{2,\parallel}
\end{equation*}
Si consideri ora un cilindretto di superficie $\Delta\Sigma$:
\begin{equation*}
    \Phi(\vb{D})= D_{2,\perp}\Delta\Sigma-D_{1,\perp}\Delta\Sigma = Q_{free}\implies K_2E_{2,\perp} -K_1E_{\,\perp}= \frac{\sigma_{free}}{\epsilon_0}
\end{equation*}

\subsubsection[Stima di chi]{Stima di $\chi_E$}
Consideriamo un material poco denso, come un gas rarefatto, in cui le interazioni tra i vari atomi sono trascurabili. 
Il momento di dipolo di ogni singolo atomo possiamo ipotizzare segua un andamento lineare al campo elettrico
\begin{equation*}
    \vb{p}_{atomo}\simeq\vb{E}\epsilon_0\alpha_E
\end{equation*}
Dove definiamo $\alpha_E$ \textit{polarizzabilità atomica} ed è legata alle caratteristiche dell'atomo
\begin{equation*}
    \vb{P}= \frac{\dd{p}}{\dd{\tau}}= n\vb{p}= n\epsilon_0\alpha_E\vb{E}\implies \chi_E = n\alpha_E 
\end{equation*}
dove $n$ è il numero di atomi per unità di volume.
gli effetti del campo locale, cioè il campo effettivamente sentito da ciascun atomo, che differisce dal campo macroscopico applicato. In questo caso si ottiene la relazione di Clausius-Mossotti.

\begin{proposition}
(Relazione di Clausius-Mossotti)
\[
\frac{\kappa_E - 1}{\kappa_E + 2} = \frac{n \alpha_E}{3}
\]
dove $\kappa_E = 1 + \chi_E$ è la costante dielettrica relativa, $n$ la densità numerica e $\alpha_E$ la polarizzabilità atomica.
\end{proposition}
\begin{proof}
Consideriamo un dielettrico isotropo costituito da atomi non interagenti immersi in un campo elettrico esterno $\vb{E}_0$. 
L'atomo si trova in una cavità sferica all'interno del materiale. Il campo locale $\vb{E}_{loc}$ sentito dall'atomo è dato da:
\[
\vb{E}_{loc} = \vb{E}_0 + \frac{\vb{P}}{3\epsilon_0}
\]
dove $\vb{P}$ è il vettore polarizzazione macroscopico.

La polarizzazione indotta su ciascun atomo è:
\[
\vb{p} = \epsilon_0 \alpha_E \vb{E}_{loc}
\]
La polarizzazione totale è:
\[
\vb{P} = n \vb{p} = n \epsilon_0 \alpha_E \vb{E}_{loc}
= n \epsilon_0 \alpha_E \left( \vb{E}_0 + \frac{\vb{P}}{3\epsilon_0} \right)
\]
\[
\vb{P} = \frac{n \epsilon_0 \alpha_E}{1 - \frac{n \alpha_E}{3}} \vb{E}_0
\]
Ma per definizione $\vb{P} = \epsilon_0 \chi_E \vb{E}_0$, quindi:

\[
\chi_E = \frac{n \alpha_E}{1 - \frac{n \alpha_E}{3}}
\]
Ricordando che $\kappa_E = 1 + \chi_E$:
\[
\implies \frac{\kappa_E - 1}{\kappa_E + 2} = \frac{n \alpha_E}{3}
\]
\end{proof}


\subsection{Esperimento di Millikan}
% da fare






\section{Elettrodinamica}

\begin{definition}
    Densità di corrente $\vb{j}$
    \begin{equation*}
        \vb{j}\cdot\dd{\Sigma}= \dv{q}{t}
    \end{equation*}
\end{definition}
Prendendo in esame un volume cilindrico:
\begin{equation*}
    \delta q = \rho\,\delta\Sigma\,\delta s \text{  oppure  } =  n q \,\delta\Sigma\,\delta s
\end{equation*}
Derivando nel tempo otteniamo:
\begin{equation*}
    \vb{j}= \rho\vb{v} \text{  oppure  } \vb{j}= nq\vb{v}
\end{equation*}
\begin{definition}
    Intensità di corrente
    \begin{equation*}
        I = \int_\Sigma \vb{j}\cdot\dd{\Sigma}
    \end{equation*}
\end{definition}

\begin{theorem}
    Principio di conservazione della carica
    \begin{equation*}
        \div(\vb{j})+\pdv{\rho}{t}=0
    \end{equation*}
\end{theorem}
\begin{proof}
    Si consideri una superficie chiusa intorno a un volume $\tau$ fisso:
    \begin{equation*}
        q_{int}= \int_\tau\rho\dd{\tau} \quad \dv{q_{int}}{t}= \dv{}{t}\int_\tau \rho\dd{\tau} = \int_\tau \pdv{\rho}{t}\dd{\tau}
    \end{equation*}
    \begin{equation*}
        \dv{q_{int}}{t}= - \oint_\Sigma \vb{j}\cdot\dd{\Sigma}\overset{Gauss}{=}-\int_\tau \div(\vb{j})\dd{\tau}\implies
        \int_\tau \pdv{\rho}{t} + \div(\vb{j}) \dd{\tau}= 0
    \end{equation*}
    \begin{equation*}
        \overset{\forall\tau}{\implies} \div(\vb{j})+ \pdv{\rho}{t}=0
    \end{equation*}

\end{proof}


\subsection{Generatori di tensione}
I generatori di tensione permettono di mantenere un circuito a tensione costante.
La circuitazione del campo elettrico lungo il circuito, in tal caso, è diverso da 0
\begin{equation*}
    \oint_\Gamma \vb{E}\cdot\dd{s}= V(\textit{fine})-V(\text{inizio}) = \mathcal{E}\quad \textit{      Forza ElettroMotrice FEM}
\end{equation*} 



\section{Magnetostatica}
Studiamo ora i fenomeni di magnetismo. Le forze magnetiche apparentemente indipendenti dai fenomeni elettrici agiscono su cariche in movimento.
Oersted scopre che la corrente produce un campo magnetico.
Si nota che le linee del campo magnetico sono tutte chiuse:
\begin{equation*}
    \oint_\Sigma \vb{B}\cdot\dd{\Sigma} = 0 \overset{Gauss}{=} \int_\tau \div(\vb{B}) \dd{\tau} \overset{\forall \tau}{\implies} \div(\vb{B})= 0
\end{equation*}
$\vb{B}$ è un campo solenoidale.

\newtheorem*{FL}{Forza di Lorentz}
\begin{FL}
    \begin{equation*}
        \vb{F}= q \left( \vb{E} \vb{v}\times \vb{B} \right)
    \end{equation*}
\end{FL}

Da questo si ricava:
\begin{equation*}
    \dd{W}= \vb{F}\cdot\dd{\vb{s}}= q\left( \vb{v}\times \right) \cdot \dd{t} = 0
\end{equation*}
La forza magnetica non compie lavoro, mantiene la velocità costante.
Una velocità perpendicolare a $\vb{B}$ genera un moto circolare uniforme.
\begin{equation*}
    m\dv{\vb{v}}{t}= q\vb{v}\times \vb{B}\qquad\qquad \vb{v}= \left( v_x,\,v_y,\,0 \right)\quad \vb{B}= \left( 0,\,0,\,B \right)
\end{equation*}

\begin{equation*}
    m\dv{v_x}{t}=qv_y\vb{B}; \quad m\dv{v_y}{t}= -qv_xB; \quad\tilde{v}= v_x + iv_y \implies m\dv{\tilde{v}}{t}= -iqB\tilde{v}
\end{equation*}
\begin{equation*}
    \tilde{v}= v_0e^{-i\left( \frac{qB}{m}t+\varphi \right)} \qquad \omega= \frac{qB}{m}\quad \textit{ pulsazione ciclotrone}
\end{equation*}
Se la componente di $\vb{v}$ parallela a $\vb{B}$ non è uguale a 0, sovrapponendo i moti ottengo un moto elicoidale.

\subsection{Effetto Hall}
















\end{document}